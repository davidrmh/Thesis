\documentclass{book}

\usepackage[utf8]{inputenc}

\usepackage[spanish]{babel}

\usepackage{graphicx}

\usepackage{mathptmx}

\begin{document}

\thispagestyle{empty}

\frontmatter

\begin{minipage}{.3\textwidth}
  \flushleft
  \center{\includegraphics[scale=0.15]{logo_ipn.pdf}}

  \vspace{2pt}

  \center{
    \rule{.5pt}{.6\textheight}
    \hspace{7pt}
    \rule{2pt}{.6\textheight}
    \hspace{7pt}
    \rule{.5pt}{.6\textheight}
  } \\

  \center{\includegraphics[scale=.24]{logo_cic_2.png}}
\end{minipage}
\begin{minipage}{.7\textwidth}
\flushright

\center{

  \center{
	\LARGE{Instituto Politécnico Nacional}
  } \\
  \rule{\textwidth}{2pt}
  \\
  \hrulefill\\[1cm]
  
  \Large{Centro de Investigación en Computación}\\[2cm]

  \large{
Aprendizaje de reglas para operaciones en el mercado accionario  }\\[2cm]

  \huge{
T \hspace{1cm} E \hspace{1cm} S \hspace{1cm} I \hspace{1cm} S  }\\[1cm]

  \large{QUE PARA OBTENER EL TÍTULO DE:}\\[1cm]

  \large{
Maestro en Ciencias de la Computación  }\\[1cm]

  \large{PRESENTA:}\\[1cm]

  \large{
David Ricardo Montalván Hernández  }\\[1cm]

  \large{
TUTOR  }\\[1cm]

  \large{
Salvador Godoy Calderón  }
}

\end{minipage}

% Aquí empieza tu tesis

\end{document}


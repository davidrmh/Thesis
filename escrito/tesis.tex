\documentclass[onesided, 12pt]{scrbook}
\usepackage[utf8]{inputenc}
\usepackage{amsmath}
\usepackage{amsthm} %Para definir ambientes con \newtheorem
\usepackage{amsfonts}
\usepackage{amssymb}
\usepackage{makeidx}
\usepackage{graphicx}
\title{Aprendizaje de reglas en el mercado accionario mexicano}
\publishers{Centro de Investigación en Computación, Instituto Politécnico Nacional}
\date{}
\author{David Ricardo Montalván Hernández}

%=========Define los ambientes a utilizar =======%
%Define estilo para dar un salto de línea en el encabezado
%del 'teorema'
\newtheoremstyle{break}
{2ex} %above space
{2ex} %below space
{\itshape} %Body font)
{} %indent amount
{\bfseries} %head font
{:} %post head puncuation
{\newline} %post head space
{}

\theoremstyle{break}
%Definición
\newtheorem{definicion}{Definicion}[chapter]

%Teorema
\newtheorem{teorema}{Teorema}[chapter]

%Algoritmo (Utiliza el ambiente tabbing)

\newtheorem{algoritmo}{Algoritmo}[chapter]
%=================================================%


\begin{document}
\maketitle
\pagenumbering{Roman} %numeración romana con mayúsculas
\renewcommand{\contentsname}{Contenido}
\tableofcontents
\renewcommand{\listfigurename}{Lista de imágenes}
\listoffigures
\renewcommand{\listtablename}{Lista de tablas}
\listoftables
\chapter*{Dedicatoria}
\chapter*{Agradecimientos}
\chapter*{Resumen}

\pagenumbering{arabic} %Numeración árabe

%=============== INTRODUCCIÓN ================= %
\chapter{Introducción}
\begin{itemize}
\item Motivación y ¿qué es lo que se busca con este trabajo?
\item Objetivo general y objetivos particulares.
\item Estructura del trabajo.
\end{itemize}

%=============== Antecedentes ================= %
\chapter{Antecedentes}
\begin{itemize}
\item Explicación de los artículos (en forma cronológica).
\end{itemize}

%=============== Mercado accionario ================= %
\chapter{El mercado accionario}

\section{Mecánica de un mercado accionario.}
\section{Estrategia Buy and Hold.}
\section{Costos de transacción}
\section{NAFTRAC e IPC.}


%=============== AQ y CN2 ================= %
\chapter{Algoritmos AQ y CN2}
\section{Algoritmo AQ}
\section{Algoritmo CN2}

%=============== Metodología propuesta ================= %
\chapter{Metodología propuesta y resultados}
\section{Separación de datos}
\section{Proceso de etiquemiento}
\section{Aprendizaje incremental}
\section{Resultados}

%=============== Conclusión ================= %
\chapter{Conclusiones y trabajo futuro}
\section{Conclusiones}
\section{Trabajo futuro}

\chapter{Bibliografía}





\end{document}
\documentclass[onesided, 12pt]{scrbook}
\usepackage[utf8]{inputenc}
\usepackage{amsmath}
\usepackage{amsthm} %Para definir ambientes con \newtheorem
\usepackage{amsfonts}
\usepackage{amssymb}
\usepackage{makeidx}
\usepackage{graphicx}
\title{Aprendizaje de reglas en el mercado accionario mexicano}
\publishers{Centro de Investigación en Computación, Instituto Politécnico Nacional}
\date{}
\author{David Ricardo Montalván Hernández}

%=========Define los ambientes a utilizar =======%
%Define estilo para dar un salto de línea en el encabezado
%del 'teorema'
\newtheoremstyle{break}
{2ex} %above space
{2ex} %below space
{\itshape} %Body font)
{} %indent amount
{\bfseries} %head font
{:} %post head puncuation
{\newline} %post head space
{}

\theoremstyle{break}
%Definición
\newtheorem{definicion}{Definicion}[chapter]

%Teorema
\newtheorem{teorema}{Teorema}[chapter]

%Algoritmo (Utiliza el ambiente tabbing)

\newtheorem{algoritmo}{Algoritmo}[chapter]
%=================================================%


\begin{document}
\maketitle
\pagenumbering{Roman} %numeración romana con mayúsculas
\renewcommand{\contentsname}{Contenido}
\tableofcontents
\renewcommand{\listfigurename}{Lista de imágenes}
\listoffigures
\renewcommand{\listtablename}{Lista de tablas}
\listoftables
\chapter*{Dedicatoria}
\chapter*{Agradecimientos}
\chapter*{Resumen}
\chapter*{Abstract}

\pagenumbering{arabic} %Numeración árabe

%=============== INTRODUCCIÓN ================= %
\chapter{Introducción}
\label{capitulo:introduccion}
%Motivación y ¿qué es lo que se busca con este trabajo?
%Objetivo general y objetivos particulares.
%Estructura del trabajo.
El incremento en el poder de cómputo, la digitalización de los mercados financieros (en particular el mercado accionario) y la oportunidad de ganar dinero, son algunos de los factores que han motivado la investigación y desarrollo de algoritmos computacionales enfocados a guiar la toma de decisiones de inversión, en particular, determinar los momentos adecuados para realizar compras o ventas.

A pesar de que la idea básica es comprar barato y vender caro, la incertidumbre y complejidad de los mercados financieros han dado lugar al uso de herramientas computacionales con el fin de guiar la toma de decisiones, concretamente, técnicas relacionadas a la inteligencia artificial han ganado notoriedad.

El objetivo general de este trabajo es proponer, bajo un paradigma simbólico, una metodología para aprender un conjunto de reglas \textit{IF..THEN}. Estas reglas se utilizarán para obtener estrategias de inversión cuya ganancia buscamos sea superior a la ganancia generada por la estrategia de \textbf{compra y espera} (ver sección \ref{seccion:hipotesis mercado eficiente}).

Una de las contribuciones de este trabajo es el análisis del mercado accionario mexicano, el cual está representado por el instrumento llamado \textbf{NAFTRAC ISHRS} (ver sección \ref{seccion:naftrac}).

Además, la metodología propuesta busca obtener un modelo interpretable, contrastando con la tendencia actual que se caracteriza por el uso de las llamadas \textit{cajas negras}, e.g., redes neuronales (ver capítulo \ref{capitulo:antecedentes}).

Los objetivos particulares de este trabajo son:
\begin{itemize}
\item Selección de atributos.

\item Etiquetado de datos.
\end{itemize}




%=============== Antecedentes ================= %
\chapter{Antecedentes}
\label{capitulo:antecedentes}
\begin{itemize}
\item Explicación de los artículos (en forma cronológica).
\end{itemize}

%=============== Marco teórico ================= %
\chapter{Marco teórico}
\label{capitulo:marco teorico}

\section{Mecánica de un mercado accionario.}
\label{seccion:mecanica del mercado}

\section{Hipótesis del mercado eficiente.}
\label{seccion:hipotesis mercado eficiente}

\section{Costos de transacción}
\label{seccion:costos de transaccion}

\section{NAFTRAC e IPC.}
\label{seccion:naftrac}

\section{Análisis técnico}
\label{seccion:analisisTecnico}

\section{Supuestos del trabajo}
\label{seccion:supuestos}


%=============== AQ y CN2 ================= %
\section{Algoritmos AQ y CN2}
\label{seccion:algoritmos aq cn2}

\subsection{Algoritmo AQ}
\label{subseccion:algoritmo aq}

\subsection{Algoritmo CN2}
\label{subseccion:algoritmo cn2}

%=============== Solución propuesta ================= %
\chapter{Solución propuesta}
\label{capitulo:solucion propuesta}

\section{Separación de datos}
\label{seccion:separacion de datos}

\section{Proceso de etiquetado}
\label{seccion:proceso etiquetado}

\section{Aprendizaje incremental}
\label{seccion:aprendizaje incremental}

%=============== Resultados ================= %
\chapter{Resultados experimentales}
\label{capitulo:resultados experimentales}

%=============== Conclusión ================= %
\chapter{Conclusiones y trabajo futuro}
\label{capitulo:conclusiones}

\section{Conclusiones}
\label{seccion:conclusiones}

\section{Trabajo futuro}
\label{seccion:trabajo futuro}

\chapter{Bibliografía}





\end{document}
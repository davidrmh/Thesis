\documentclass[12pt]{scrbook}
\usepackage[utf8]{inputenc}
\usepackage{amsmath}
\usepackage{amsthm} %Para definir ambientes con \newtheorem
\usepackage{amsfonts}
\usepackage{amssymb}
\usepackage{makeidx}
\usepackage{graphicx}
\usepackage{natbib}
\usepackage{url}
%\usepackage{float}

\title{Aprendizaje automático de reglas para operar en los mercados de índices accionarios}
\publishers{Centro de Investigación en Computación, Instituto Politécnico Nacional}
\date{}
\author{David Ricardo Montalván Hernández}

%=========Define los ambientes a utilizar =======%
%Define estilo para dar un salto de línea en el encabezado
%del 'teorema'
\newtheoremstyle{break}
{2ex} %above space
{2ex} %below space
{\itshape} %Body font)
{} %indent amount
{\bfseries} %head font
{:} %post head puncuation
{\newline} %post head space
{}

\theoremstyle{break}
%Definición
\newtheorem{definicion}{Definición}[chapter]

%Teorema
\newtheorem{teorema}{Teorema}[chapter]

%Notas importantes
\newtheorem{nota}{Nota}[chapter]

%Algoritmo (Utiliza el ambiente tabbing)
\theoremstyle{break}
\newtheorem{algoritmo}{Algoritmo}[chapter]
%=================================================%

%====================Macros=======================%
\newcommand{\buyhold}{\textit{buy-and-hold} }
%=================================================%

\begin{document}
\maketitle
\pagenumbering{Roman} %numeración romana con mayúsculas
\renewcommand{\contentsname}{Contenido}
\tableofcontents
\renewcommand{\listfigurename}{Lista de imágenes}
\listoffigures
\renewcommand{\listtablename}{Lista de tablas}
\renewcommand\tablename{Tabla}
\renewcommand{\bibname}{Referencias}
\renewcommand{\figurename}{Imagen}
\listoftables

\chapter*{Dedicatoria}
A mis padres y amigos, quienes seguramente se reirán de esta dedicatoria y su "gran emotividad".

\chapter*{Agradecimientos}
Agradezco a mi director de tesis el doctor Salvador Godoy Calderón por enseñarme el enfoque simbólico del aprendizaje de máquina y por tener la paciencia suficiente para  intentar librarme (desafortunadamente sin éxito) de mis "manías estadísticas".

Agradezco también a cada uno de los miembros de mi comité tutorial por sus valiosas observaciones.

\chapter*{Resumen}
\chapter*{Abstract}

\pagenumbering{arabic} %Numeración árabe

%=============== INTRODUCCIÓN ================= %
\chapter{Introducción}
\label{capitulo:introduccion}
%Motivación y ¿qué es lo que se busca con este trabajo?
%Objetivo general y objetivos particulares.
%Estructura del trabajo.
El incremento en el poder de cómputo, la digitalización de los mercados financieros (en particular el mercado accionario) y la oportunidad de ganar dinero, son algunos de los factores que han motivado la investigación y desarrollo de algoritmos computacionales enfocados a guiar la toma de decisiones de inversión, en particular, determinar los momentos adecuados para realizar compras o ventas.

A pesar de que la idea básica es comprar barato y vender caro, la incertidumbre y complejidad de los mercados financieros han dado lugar al uso de herramientas computacionales con el fin de guiar la toma de decisiones, concretamente, técnicas relacionadas a la inteligencia artificial han ganado notoriedad.

El objetivo general de este trabajo es proponer una metodología para aprender, de manera automática, un conjunto de reglas \textit{IF..THEN}. Estas reglas se utilizarán para obtener estrategias de inversión cuya ganancia, buscamos, sea superior a la ganancia generada por la estrategia de \textbf{compra y espera} (ver sección \ref{seccion:mercado eficiente}).

Una de las contribuciones de este trabajo es el análisis del mercado accionario mexicano, el cual está representado por el instrumento llamado \textbf{NAFTRAC ISHRS} (ver sección \ref{seccion:indices accionarios}).

Además, la metodología propuesta busca obtener un modelo interpretable, contrastando con la tendencia actual que se caracteriza por el uso de las llamadas \textit{cajas negras}, e.g., redes neuronales (ver capítulo \ref{capitulo:antecedentes}).

Los objetivos particulares de este trabajo son:
\begin{itemize}
\item Selección de atributos.

\item Etiquetado de datos.
\end{itemize}




%=============== Antecedentes ================= %
\chapter{Antecedentes}
\label{capitulo:antecedentes}
\begin{itemize}
\item Explicación de los artículos (en forma cronológica).
\end{itemize}

%=============== Marco teórico ================= %
\chapter{Marco teórico}
\label{capitulo:marco teorico}

\section{Mercado de índices accionarios}
\label{seccion:indices accionarios}

Debido a la diversidad de los tipos de mercados financieros (, deuda, tipo de cambio, derivados, etc.) así como al funcionamiento particular de cada uno de ellos, fue necesario restringir este estudio a un mercado en particular. Siendo concretos, este trabajo únicamente analiza el mercado de índices .

Un índice accionario, busca representar el comportamiento de un segmento específico del mercado accionario, por ejemplo un sector o un área geográfica en particular \cite{CFA2019-market-index}.

Así pues, el propósito principal de este tipo de índices es representar la opinión colectiva que se tiene respecto a un conjunto de acciones.

En este trabajo consideraremos lo siguientes índices :

\begin{itemize}
\item Índice de precios y cotizaciones (S\&P/BMV IPC)
\item Índice Standard and Poors 500 (S\&P 500)
\end{itemize}

El índice de precios y cotizaciones es el principal indicador del mercado accionario mexicano. Este índice busca medir el desempeño de las acciones listadas en la Bolsa Mexicana de Valores utilizando una muestra con las 35 series accionarias de mayor tamaño y liquidez listadas en dicha bolsa.\footnote{https://espanol.spindices.com/indices/equity/sp-bmv-ipc}

Por otro lado, el índice Standard and Poors 500 es considerado como el mejor indicador para el mercado accionario de Estados Unidos. Este índice está compuesto por las 500 empresas de mayor capitalización las cuales capturan alrededor del $80\%$ de la capitalización del mercado\footnote{https://us.spindices.com/indices/equity/sp-500}

\begin{nota} \label{nota:ETF}
En la práctica los inversionistas no pueden comprar los índices accionarios, en cambio, los instrumentos negociados reciben el nombre de Títulos Referenciados a Acciones (TRACS\footnote{En inglés Exchange Traded Funds o ETF}). Estos instrumentos buscan replicar el comportamiento de las acciones o portafolio al que están referenciados.

En nuestro caso, el instrumento que busca replicar el índice de precios y cotizaciones recibe el nombre de iShares NAFTRAC (o simplemente NAFTRAC), mientras que para el índice Standard and Poors 500 el instrumento relacionado es el SPDR S\&P 500.
\end{nota}

\section{Mercado eficiente}
\label{seccion:mercado eficiente}
Cuando hablamos de un mercado eficiente, nos referimos a un mercado que es capaz de capturar, de forma rápida y racional, la información disponible más reciente. Esta información es reflejada en los precios de los instrumentos financieros y como consecuencia, en un mercado eficiente, los precios son informativos, lo que nos permite lograr una colocación eficiente de nuestros recursos \cite{CFA2019}.

\subsection{Versiones de eficiencia en el mercado}
\label{subseccion:versiones emh}
Eugene Fama\footnote{Premio Nobel en economía en 2013} describe tres formas de eficiencia de un mercado: débil, semifuerte y fuerte \cite{Fama1965}.

En la versión débil, los precios actuales reflejan toda la información histórica del mercado, es decir, la información del pasado relativa a precios y al volumen negociado. En consecuencia, bajo esta versión de eficiencia, no es posible utilizar datos históricos con el fin de predecir la tendencia futura de los precios (ver sección \ref{seccion:analisisTecnico}).

La versión semifuerte incluye la versión débil y añade que, además de reflejar la información histórica del mercado, los precios actuales reflejan toda la información pública. Por información pública nos referimos a la información a la cual el público inversionista puede acceder, por ejemplo, noticias económicas o financieras, estados de resultados de las empresas, reportes anuales o trimestrales de las mismas, indicadores macroeconómicos etc.

De acuerdo a esta versión, no es posible utilizar la información pública con el fin de predecir la tendencia futura de los precios.

Finalmente, la versión fuerte de la eficiencia de un mercado, nos dice que los precios actuales reflejan totalmente tanto la información pública como la información privada, es decir, aquella que solamente algunas personas poseen. Por lo tanto, bajo esta versión, incluso tomando en cuenta la información privada, no es posible predecir la tendencia futura de los precios.

Observamos entonces que, la eficiencia de un mercado debe ser interpretada relativa al tipo de información que los inversionistas utilizan. Además, como se señala en \cite{CFA2019}, la eficiencia cae en un espectro continuo, que varía a través del tiempo, regiones geográficas y tipo de mercado.

La imagen \ref{imagen:versiones emh} ilustra las distintas versiones de la eficiencia de un mercado.

En este trabajo nos enfocamos únicamente en la versión débil.

\begin{figure}[ht]
\centering
\scalebox{0.8}{\includegraphics[width=.8\linewidth]{imagenes/versiones-emh.jpeg}}
\caption{\label{imagen:versiones emh} Versiones de la eficiencia de un mercado}
\end{figure}


\subsection{Estrategia \buyhold}
\label{seccion:buy and hold}
Como se señala en la sección \ref{subseccion:versiones emh}, de acuerdo a la hipótesis del mercado eficiente, no es posible utilizar información histórica con el fin de predecir los precios del futuro. Lo mejor que uno puede hacer es comprar un portafolio bien diversificado y mantenerlo por un periodo de tiempo predeterminado, esta estrategia recibe el nombre de estrategia \buyhold.

De acuerdo a lo anterior tenemos entonces que, en un mercado eficiente, no es posible obtener, de manera consistente y sistemática, ganancias superiores a aquellas generadas por la estrategia \buyhold.

Dado un plazo de tiempo fijo, $\left[T_{I}, T_{F}\right]$, la ganancia de la estrategia \buyhold está dada por la ecuación (\ref{eqn:ganancia BH})

\begin{equation} \label{eqn:ganancia BH}
G_{\left[T_{I}, T_{F}\right]} (BH) = \dfrac{P_{T_F} (1 - c) } { P_{T_I} (1 + c) } - 1
\end{equation}

en donde $P_{T_F}$ es el precio de ejecución al final del plazo, $P_{T_I}$ es el precio de ejecución al inicio de este y $c$ es el costo (porcentual) de cada transacción (ver sección \ref{sec:supuestos del mercado}). 

La imagen \ref{imagen:buy hold alza} ejemplifica la estrategia \buyhold en un periodo en el cual el mercado presenta una tendencia a la alza.\footnote{En la jerga financiera, un mercado bullish} Como podemos observar, vencer esta estrategia cuando se presenta este tipo de tendencia es todo un reto, esto se debe a que sólo se involucran dos operaciones, lo que implica un ahorro en costos de transacción.

\begin{figure}[ht]
\centering
\scalebox{0.8}{\includegraphics[width=1\linewidth]{imagenes/buy-hold-alza.jpeg}}
\caption{\label{imagen:buy hold alza} Estrategia \buyhold en un mercado a la alza}
\end{figure}

\subsection{Exceso de ganancia sobre \buyhold}

Para comparar las estrategias propuestas en este trabajo contra la estrategia \buyhold, estaremos utilizando como métrica de desempeño el exceso de ganancia (porcentual).

Esta métrica compara la ganancia obtenida por una estrategia, $S$, con aquella obtenida por la estrategia \buyhold.

Para un plazo fijo, $\left[T_{I}, T_{F}\right]$, el exceso de ganancia de una estrategia $S$, sobre la estrategia \buyhold, está dado por la ecuación (\ref{eqn:Exceso de ganancia})

\begin{equation} \label{eqn:Exceso de ganancia}
ExG_{\left[T_{I}, T_{F}\right]} = G_{\left[T_{I}, T_{F}\right]} (S) - G_{\left[T_{I}, T_{F}\right]} (BH)
\end{equation}

en donde la ganancia $G_{\left[T_{I}, T_{F}\right]} (BH)$, está dada por la ecuación (\ref{eqn:ganancia BH}) y la ganancia $G_{\left[T_{I}, T_{F}\right]} (S)$ es simplemente la variación porcentual entre el capital inicial disponible en el momento $T_{I}$ y el capital con el que se cuenta en el momento $T_{F}$.

\begin{equation} \label{eqn:Ganancia estrategia S}
G_{\left[T_{I}, T_{F}\right]}(S) = \dfrac{Capital_{T_F}}{Capital_{T_I} } - 1 
\end{equation}

Claramente buscamos que $ExG_{\left[T_{I}, T_{F}\right]}$ sea un número positivo.

\section{Análisis técnico}
\label{seccion:analisisTecnico}
Contrario a lo que establece la hipótesis del mercado eficiente (sección \ref{seccion:mercado eficiente}), el análisis técnico busca predecir la tendencia futura en los precios utilizando la información histórica del mercado (precios y volumen) \cite{murphy1999technical}.

Los tres supuestos en los que se basa este tipo de análisis son los siguientes:

\begin{itemize}
\item La información de todos los factores que afectan a la oferta y a la demanda se encuentra reflejada en el precio de los instrumentos financieros. En consecuencia, basta centrarse en el estudio de los precios para poder analizar el mercado.

\item Los precios se mueven siguiendo tendencias (a la alza o a la baja). El objetivo del análisis técnico es detectar, lo más pronto posible, el inicio y el final de estas tendencias.

\item Los patrones en los precios se repiten a lo largo del tiempo.
\end{itemize}


\subsection{Indicadores técnicos}
\label{subseccion:indicadores tecnicos}
Para realizar el pronóstico de las tendencias, el análisis técnico se vale del uso de indicadores técnicos. Estos indicadores son transformaciones aplicadas a la información histórica para una ventana (deslizante) de tiempo determinada \cite{murphy1999technical}, \cite{technicalAnalysisKirkPatrick}, \cite{encycoplediaTechnicalIndicators}.

Los indicadores técnicos utilizados en este trabajo, serán\footnote{Aunque en las fórmulas no se indica, cada indicador se calcula en cada periodo $t$ utilizando la información en el intervalo $\left[t-n + 1, t\right]$, es decir, la información contenida en una ventana deslizante de tamaño $n$. }:

\begin{itemize}
\item Oscilador aroon.

\item Relative strength index (RSI).

\item Money flow index (MFI).

\item Williams \%R.

\item Commodity channel index (CCI).
\end{itemize}


\subsubsection{Oscilador aroon}
\label{subsubseccion:Oscilador Aroon}
Este indicador mide el número de periodos, dentro de una ventana de tiempo de tamaño $n$, que han transcurrido desde el máximo y el mínimo más reciente. El cálculo de este indicador está dado en la ecuación (\ref{eqn:oscilador aroon})

\begin{equation} \label{eqn:oscilador aroon}
OsAroon = 100 \left( \dfrac{  n - T_H  } { n } - \dfrac{  n - T_L  } { n } \right)
\end{equation}

en donde $n$ es el número de periodos que comprende la ventana de tiempo, $T_H$ es el número de periodos transcurridos desde el último máximo registrado dentro de la ventana de tamaño $n$ y $T_L$ es el número de periodos transcurridos desde el último mínimo registrado dentro de la ventana de tamaño $n$.

Vemos entonces que, de acuerdo a la ecuación (\ref{eqn:oscilador aroon}), si $OsAroon$ es un número positivo, los precios muestran una tendencia a la alza (debemos de prepararnos para comprar). En cambio cuando $OsAroon$ es un número negativo, los precios muestran una tendencia a la baja (debemos de prepararnos para vender).

\subsubsection{Relative strength index (RSI)}
\label{subsubseccion:RSI}
Este indicador busca predecir la tendencia en los precios analizando el promedio de ganancias y pérdidas obtenidas para una ventana de tiempo de tamaño $n$. La fórmula para calcular el RSI está dada en la ecuación (\ref{eqn:RSI})

\begin{equation} \label{eqn:RSI}
RSI = 100 - \left( \frac{100}{1 + RS} \right)
\end{equation}

en donde 

\begin{equation} \label{eqn:RSI RS}
RS = \frac{Ganancia\,promedio\,en\,n\,periodos}{P\acute{e}rdida\,promedio\,en\,n\,periodos}
\end{equation}

Típicamente, en la literatura financiera (\cite{technicalAnalysisKirkPatrick}, \cite{encycoplediaTechnicalIndicators}) una señal de venta se genera cuando $RSI > 70$; por otra parte, una señal de compra se genera si $RSI < 30$.

\subsubsection{Money flow index (MFI)}
\label{subsubseccion:money flow index}
Este indicador es similar al RSI. Considera el número de periodos en los que se han presentado ganancias o pérdidas tomando en cuenta tanto los precios como el volumen del periodo (en comparación con el RSI que sólo considera los precios).

Para el cálculo del MFI se necesita primero calcular el precio promedio del periodo $t$, el cual está dado por la ecuación (\ref{eqn:precio tipico})

\begin{equation} \label{eqn:precio tipico}
Precio\, \, promedio_t = \dfrac{H_{t} + L_{t} + C_{t}}{3}
\end{equation}

en donde $H_{t}, L_t, C_t$ representan, respectivamente, el precio máximo, mínimo y de cierre en el periodo $t$.

Una vez calculado el precio promedio, se calcula el flujo de efectivo del periodo, como se muestra en la ecuación (\ref{eqn:flujo de efectivo})

\begin{equation} \label{eqn:flujo de efectivo}
MF_t = Precio\, \, promedio_t \times Volumen_t \times F_t
\end{equation}

en donde $F_t = 1$, si el precio promedio del periodo $t$ es mayor al precio promedio del periodo $t-1$, y $F_t = -1$ en el caso contrario.

Así, dada una ventana de tiempo de tamaño $n$, el MFI se calcula de acuerdo a la ecuación (\ref{eqn:MFI})

\begin{equation} \label{eqn:MFI}
MFI = 100 - \left( \frac{100}{1 + MFR} \right)
\end{equation}

en donde $MFR$ está dado por 

\begin{equation} \label{eqn:MFR}
MFR = \dfrac{\sum_{MF_t > 0} MF_t  }{\sum_{MF_t < 0} MF_t}
\end{equation}

con $t$ dentro del periodo de la ventana $n$.

En la práctica, es común considerar que una señal de venta se da cuando $MFI > 80$, mientras que una señal de compra se da cuando $MFI < 20$.

\subsubsection{Williams \%R}
\label{subsubseccion:Williams R}
Este indicador busca reflejar el precio de cierre relativo al precio máximo dentro de una ventana de tiempo dada.

Su cálculo está dado por la ecuación (\ref{eqn:williams R})

\begin{equation} \label{eqn:williams R}
Williams \%R_t = -100 \times \dfrac{HH - C_t}{HH - LL}
\end{equation}

en donde $HH$ es el máximo entre los precios máximos de cada periodo que comprende la ventana de tiempo, de manera similar $LL$ es el mínimo de los precios mínimos de cada periodo en la ventana de tiempo y $C_t$ es el precio de cierre en el tiempo $t$.

Se considera típicamente que si este indicador se encuentra entre $0$ y $-20$, es un momento apropiado para vender, mientras que si se encuentra dentro del rango de $-80$ a $-100$, es un momento adecuado para comprar.

%=============== AQ y CN2 ================= %
\section{Algoritmos AQ y CN2}
\label{seccion:algoritmos aq cn2}

\subsection{Algoritmo AQ}
\label{subseccion:algoritmo aq}
El algoritmo cuasi-óptimo (AQ), es un algoritmo de aprendizaje supervisado, que induce un conjunto de reglas del tipo \textit{Si-entonces} (ver \cite{AQCervone2010}, \cite{AQMichalski1991} y \cite{AQWojtusiak2012} para un estudio detallado). 

Dado un conjunto, $P$, con $n$ observaciones, $P_1, P_2, \ldots, P_n$ (ejemplos de la clase positiva)  y un conjunto, $N$, con $m$ observaciones, $Q_1, Q_2, \ldots, Q_m$ (ejemplos de la clase negativa), el algoritmo AQ encuentra un conjunto de reglas que son completas (cubren todos los ejemplos de la clase positiva) y consistentes (no cubren ejemplos de la clase negativa).

Las reglas toman la forma descrita en ecuación (\ref{eqn:reglas AQ})

\begin{equation} \label{eqn:reglas AQ}
Premisa \rightarrow Consecuente
\end{equation}

en donde $Premisa$ es una conjunción de selectores (esta conjunción recibe el nombre de complejo) y cada selector tiene la estructura dada en la ecuación 

\begin{equation} \label{eqn:condicion AQ}
\left[Atributo\,\, OP\,\, Valores \right]
\end{equation}

en donde el término $OP$ depende del tipo de atributo que se está utilizando, por ejemplo, para un atributo con valores continuos tenemos que $OP \in \{>, \geq, <, \leq\}$.

El $Consecuente$ en (\ref{eqn:reglas AQ}), es una sola condición, por ejemplo, comprar.

El algoritmo AQ inicia su proceso de inducción seleccionando un ejemplo (semilla), $S$, de la clase positiva, el cual es generalizado creando todos los complejos que cubren $S$ y que son consistentes, es decir, que no cubren ningún ejemplo de la clase negativa. Este conjunto de complejos, $G(S,N)$, recibe el nombre de estrella. El mejor complejo en la estrella es seleccionado de acuerdo a un criterio previamente establecido, en este trabajo el criterio utilizado es maximizar el número de ejemplos positivos cubiertos. Este proceso es repetido hasta tener una disyunción de complejos (llamada cobertura) que es completa y consistente. El algoritmo \ref{algo:AQ} muestra un pseudocódigo para el algoritmo AQ.

\begin{algoritmo}[Algoritmo AQ]
\begin{tabbing}
\\$P$ el conjunto de ejemplos positivos de la clase C
\\$N$ el conjunto de ejemplos negativos de la clase C\\
1. \=$Cobertura\leftarrow \emptyset $ \\
2. Mientras $P \neq \emptyset$:\\
 \>3. Elige semilla $S$ en $P$\\
 \>4. Genera estrella $G(S,N)$\\
 \>5. Selecciona el mejor complejo, $c$, en $G(S,N)$\\
 \>6. Agrega $c$ a $Cobertura$\\
 \>7. Elimina de $P$ los ejemplos cubiertos por $c$\\
\=8. Regresa $Cobertura$
\end{tabbing}
\label{algo:AQ}
\end{algoritmo}

\begin{algoritmo}[Genera estrella]
\begin{tabbing}
\\Sea $S$ la semilla correspondiente a un ejemplo en la clase positiva.
\\Sea $N$ el conjunto de ejemplos negativos de la clase C\\
1. \=$G(S,N)\leftarrow \emptyset$ (complejo vacío cubre todas las observaciones) \\
2. Mientras $G(S,N)$ cubra algún ejemplo de $N$:\\
 \>3. Elige un ejemplo negativo $E_{neg}$ cubierto por $G(S,N)$\\
 \>4. \= Especializa los complejos en $G(S,N)$ con el fin de excluir $E_{neg}$\\
 \>\>4.1 Sea $EX$ el conjunto de selectores que cubren $S$ pero no $E_{neg}$ \\
 \>\>4.2 $G(S,N)=\{x \wedge y \vert x \in G(S,N), y \in EX \}$\\
\=6. Regresa $G(S,N)$
\end{tabbing}
\label{algo:AQ genera estrella}
\end{algoritmo}

\subsection{Algoritmo CN2}
\label{subseccion:algoritmo cn2}
Como podemos observar en el algoritmo \ref{algo:AQ genera estrella}, al momento de generar la estrella, el algoritmo AQ sólo considera especializaciones que excluyen un ejemplo negativo en particular mientras que al mismo tiempo se busca cubrir la semilla. Como se señala en \cite{CN2-Clark1989}, este tipo de búsqueda se limita al espacio de complejos que son consistentes con el conjunto de entrenamiento, limitando así la capacidad de generalización del algoritmo.

En cambio, el algoritmo CN2 remueve esta dependencia en ejemplos específicos y extiende el espacio de búsqueda al examinar todas las especializaciones posibles de un complejo.

Para encontrar el mejor complejo, el algoritmo CN2 utiliza dos heurísticas para guiar el proceso de creación de la estrella. La primera de ellas utiliza la entropía para medir la calidad de los complejos (entre menor entropía mejor es el complejo), así, dado un conjunto de ejemplos, $E$, cubiertos por el complejo $c$, si $p_i$ denota la probabilidad de la clase $i$ dentro de los ejemplos en $E$, entonces la entropía del complejo $c$ está dada por:

\begin{equation} \label{eqn:entropia complejo}
entrop(c) = - \sum_{i} p_i log_{2}(p_{i})
\end{equation}

La segunda heurística se utiliza para detectar complejos que son estadísticamente significativos. Dada una clase de ejemplos, $E$, cubiertos por un complejo $c$, esta heurística compara la distribución observada de las clases entre los ejemplos en $E$ y la distribución esperada de las clases en una muestra del mismo tamaño de $E$.

Si $f_i$ es la frecuencia observada de la clase $i$ dentro de los ejemplos en $E$, $e_{i}$ es la frecuencia esperada de la clase $i$ para una muestra de tamaño $N=\vert E \vert$, es decir $e_i = N \times Prob\left[ Clase = i \right]$+
63 y $k$ es el número de clases, entonces probamos la significancia del complejo $c$ utilizando la estadística dada en la ecuación\footnote{Esta estadística tiene una distribución $\chi^2$ con $k-1$ grados de libertad } (\ref{eqn:cn2 estadistica sig})

\begin{equation} \label{eqn:cn2 estadistica sig}
2 \sum_{i=1}^{k} f_{i} log\left(\frac{f_{i}} {e_{i}} \right)
\end{equation}

\begin{algoritmo}[Algoritmo CN2]
\begin{tabbing}
\\Sea $D$ el conjunto de datos etiquetados
\\Sea $MC$ el mejor complejo\\
1. \=$REGLAS$ $\leftarrow \emptyset $ \\
2. Mientras $D \neq \emptyset$ o $MC \neq \emptyset$:\\
 \>3. Crea estrella y selecciona el mejor complejo $MC$\\
 \>4. Sea $EC$ el conjunto de ejemplos cubiertos por $MC$ \\
 \>5. Elimina $EC$ de $D$\\
 \>6. Sea $CLA$ la clase más común en $EC$\\
 \>7. Agrega la reglas "Si $MC$ entonces $CLA$" a $REGLAS$\\
8. Regresa $REGLAS$
\end{tabbing}
\label{algo:CN2}
\end{algoritmo}

%=============== Solución propuesta ================= %
\chapter{Solución propuesta}
\label{capitulo:solucion propuesta}

\section{Separación de datos}
\label{seccion:separacion de datos}
En este trabajo se utilizan precios diarios\footnote{Obtenidos de Yahoo Finance. https://finance.yahoo.com} de los instrumentos NAFTRAC y SPDR S\&P 500 (ver sección \ref{seccion:indices accionarios}).

Para el NAFTRAC, la información inicia el día 15 de febrero de 2013 y termina el 21 de enero de 2019. Para el SPDR S\&P 500 el periodo de tiempo abarcado es del 2 de enero de 2008 hasta  el 3 de marzo de 2019.

La tabla \ref{tabla:Ejemplo datos diarios NAFTRAC} muestra la estructura de los datos para el NAFTRAC, es importante señalar que los datos no incluyen días festivos ni fines de semana.

\begin{table}[h]
\centering
\begin{tabular}{cccccccc}
\hline
\textbf{Fecha} & \textbf{Apertura} & \textbf{Máximo} & \textbf{Mínimo} & \textbf{Cierre} & \textbf{Cierre ajustado} & \textbf{Volumen} \\
\hline
2013/02/15 & 44.09 & 44.24 & 43.86 & 44.18 & 41.61 & 77,614,608\\
2013/02/18 & 44.38 & 44.38 & 44.04 & 44.16 & 41.60 & 6,457,428\\
2013/02/19 & 44.34 & 44.77 & 44.29 & 44.63 & 42.05 & 68,042,072\\
\vdots & \vdots & \vdots & \vdots & \vdots & \vdots & \vdots \\
\hline
\end{tabular}
\caption{\label{tabla:Ejemplo datos diarios NAFTRAC} Datos diarios NAFTRAC}
\end{table}

Para crear nuestros conjuntos de entrenamiento y prueba, los datos se dividen utilizando una ventana deslizante de 90 días. Este criterio está fundamentado en el hecho de que, tanto en México como en Estados Unidos, las empresas con acciones listadas en alguna bolsa de valores, están obligadas a reportar los resultados de sus operaciones de manera trimestral, así pues, buscamos capturar la reacción de los inversionistas ante la publicación de dicha información.

La tabla \ref{tabla:data split SP500} muestra la separación de los datos del SPDR S\&P 500.
\begin{table}[h]
\centering
\begin{tabular}{cc}
\hline
\textbf{Entrenamiento} & \textbf{Prueba} \\
\hline
2008/01/02 - 2008/05/09 & 2008/05/12 - 2008/09/17 \\
2008/05/12 - 2008/09/17 & 2008/09/18 - 2009/01/27 \\
2008/09/18 - 2009/01/27 & 2009/01/28 - 2009/06/05 \\
\vdots & \vdots \\
\hline
\end{tabular}
\caption{\label{tabla:data split SP500} Separación de datos}
\end{table}

Observemos que con esta forma de separar los datos, un conjunto de prueba se convierte en un nuevo conjunto de entrenamiento, esto nos permite capturar el comportamiento dinámico de los mercados, en particular, su comportamiento no estacionario.

Utilizando esta metodología, se obtienen 15 conjuntos de entrenamiento y 15 conjuntos de prueba con la información del NAFTRAC, por otra parte, con los datos del SPDR S\&P 500 se obtienen 30 conjuntos de entrenamiento y 30 conjuntos de prueba.



\section{Proceso de etiquetado}
\label{seccion:proceso etiquetado}
Ya que nuestros datos de entrenamiento no están etiquetados, utilizamos un algoritmo evolutivo para obtener un conjunto de etiquetas que nos permita entrenar cada modelo. La idea detrás del etiquetado se basa en encontrar las combinaciones de compras y ventas que nos generen la mayor ganancia una vez conocida toda la historia del periodo de entrenamiento.

Este algoritmo evolutivo pertenece a la clase de algoritmos para la estimación de distribución (EDA por sus siglas en inglés \cite{simon2013evolutionary}). Cada individuo en la población representa una estrategia de inversión para un periodo de entrenamiento dado. Esta estrategia está representada a través de un vector $\mathbf{x} = (x_1, x_2, \ldots, x_t)$ cuya componente $x_i$ representan la acción a tomar en el día $i$, $x_i \in \{-1,0,1\}$, en donde $-1$ representa una señal de venta, $0$ una señal de espera y $1$ una señal de compra. La función de aptitud utilizada fue el exceso de ganancia sobre la estrategia \buyhold.

En la imagen \ref{imagen:etiquetado} vemos el resultado del proceso de etiquetado para un conjunto de entrenamiento en particular.

\begin{figure}[ht]
\centering
\scalebox{0.8}{\includegraphics[width=1\linewidth]{imagenes/etiquetado.jpeg}}
\caption{\label{imagen:etiquetado} Resultado del proceso de etiquetado}
\end{figure}

\section{Atributos utilizados}
\label{seccion:atributos}
Los atributos utilizados en cada uno de los experimentos, son el conjunto de indicadores técnicos descritos en la sección \ref{subseccion:indicadores tecnicos}. Los parámetros utilizados para cada uno de ellos se describen a continuación.


\begin{itemize}
\item Oscilador Aroon con una ventana de tiempo de 25 días para cada cada indicador Aroon.

\item RSI con una ventana de tiempo de 14 días.

\item MFI con una ventana de tiempo de 14 días.

\item Williams \%R con una ventana de tiempo de 14 días.

\item CCI con una ventana de tiempo de 20 días y un factor $C=0.015$.
\end{itemize}


\section{Aprendizaje incremental}
\label{seccion:aprendizaje incremental}
Tanto como para el algoritmo AQ como para el algoritmo CN2 se realizaron dos tipos de experimentos. El primero de ellos está basado en un enfoque de \textit{aprender-aplicar-descartar}, en el cual, dado un conjunto de entrenamiento, se induce un conjunto de reglas las cuales se aplican al conjunto de prueba más cercano. Después de esto, las reglas aprendidas se descartan (mecanismo de olvido).

El otro experimento se basa en un \textit{aprendizaje incremental}, en este caso aprendemos un conjunto de reglas y en lugar de descartarlas, las ordenamos de acuerdo a su desempeño sobre el conjunto de entrenamiento del cual fueron inducidas por primera vez, así como su desempeño sobre los conjuntos de prueba en los cuales han sido aplicadas. En este enfoque mantenemos un conjunto de $k$-mejores reglas de compra y $k$-mejores reglas de venta.

Para medir el desempeño de las reglas sobre los conjuntos de entrenamiento de los cuales se inducen por primera vez, utilizamos la suma de su \textit{soporte} y su \textit{exactitud de Laplace}.

En cambio, para medir el desempeño de las reglas sobre los conjuntos de prueba, las agrupamos en pares de \textit{compra-venta}. Estos pares de reglas se forman a partir de las señales de compra y venta generadas durante el periodo de prueba y son recompensadas (penalizadas) de acuerdo a la ganancia (pérdida) que generan.

Por ejemplo, consideremos un caso en el cual la regla de compra, $R_{C}$, nos indica que en el tiempo $t_1$ se debe de comprar, esta compra tendrá un precio de ejecución $P_{C}$. Conforme avanza el tiempo, notamos que la regla de venta, $R_{V}$, nos indica que en el tiempo $t_2 > t_1$, se debe de vender, esta venta tendrá precio de ejecución $P_{V}$. La recompensa (penalización) que se le asigna a cada regla en el par $\left(R_C, R_V\right)$ está dada por su ganancia (pérdida) porcentual, es decir:

\begin{equation}\label{eqn:recompensa reglas}
Recompensa(R_C, R_V) = \dfrac{P_V (1 - c)}{P_C (1 + c) } - 1
\end{equation}

en donde $c$ es el costo de transacción.

Finalmente, cada regla que no aparece en el conjunto de prueba que se está evaluando, se penaliza con la pérdida promedio que se tiene sobre este mismo conjunto de prueba; esto se utiliza como un mecanismo para ir descartando reglas debido al paso del tiempo.

De acuerdo a lo anterior, el puntaje de una regla $R$, antes de evaluar el conjunto de prueba $M$, está dado por:

\begin{equation} \label{eqn:puntaje de una regla}
puntaje_{-M}(R) = soporte(R) + Laplace(R) + \sum Recompensas
\end{equation}

en la ecuación (\ref{eqn:puntaje de una regla}), el término $ \sum Recompensas$ hace referencia a la suma de recompensas o penalizaciones que ha obtenido la regla $R$ para cada conjunto de prueba anterior al conjunto $M$.

De esta manera, ahora somos capaces de ordenar las reglas y por lo tanto acumular el aprendizaje así como identificar reglas útiles y descartar aquellas que no han funcionado.

La imagen \ref{imagen:aprendizaje_incremental} muestra el esquema del aprendizaje incremental que se propone en este trabajo.


\begin{figure}[ht]
\centering
\scalebox{1.0}{\includegraphics[width=1\linewidth]{imagenes/aprendizaje-incremental.jpg}}
\caption{\label{imagen:aprendizaje_incremental} Aprendizaje incremental}
\end{figure}


\section{Límites para las señales de venta}
\label{seccion:limites ventas}
Además de las reglas inducidas por cada algoritmo, utilizamos dos reglas adicionales las cuales buscan capturar el perfil de riesgo de los inversionistas.

Estas nuevas reglas establecen un límite superior (mercados a la alza) y un límite inferior (mercados a la baja) relativos a la última compra realizada. El primero de estos límites (límite superior) puede ser interpretado como el nivel de codicia que tiene un inversionista, mientras que el segundo (límite inferior) se interpreta como el nivel de aversión al riesgo.

De esta manera, se recibe una señal de venta en el tiempo $t$ si alguna de las siguientes condiciones se cumple:

\begin{align}
R_{venta} \wedge (Var_{t} > LS) \label{eqn:Venta limite sup}\\
R_{venta} \wedge (Var_{t} < LI) \label{eqn:Venta limite inf}
\end{align}

en donde $R_{venta}$ representa una regla de venta inducida por los algoritmos AQ o CN2, $LS$ es el límite superior expresado como un porcentaje relativo al último precio de compra, $LI$ es el límite inferior expresado como un porcentaje relativo al último precio de compra y $Var_{t}$ representa la variación porcentual del precio de ejecución en el tiempo $t$, $P_{t}^{ejec}$, respecto al último precio de compra, $P_{C}^{ult}$, como se muestra en la ecuación (\ref{eqn:variacion bandas})

\begin{equation} \label{eqn:variacion bandas}
Var_{t} = \dfrac{P_{t}^{ejec}}{P_{C}^{ult}} - 1
\end{equation}

La imagen \ref{imagen:bandas horizontales} muestra un ejemplo de los límites relativos a una señal de compra.

\begin{figure}[ht]
\centering
\scalebox{0.66}{\includegraphics[width=1\linewidth]{imagenes/bandas_horizontales.jpeg}}
\caption{\label{imagen:bandas horizontales} Límites para las señales de venta relativos a una señal de compra}
\end{figure}

\section{Supuestos del mercado}
\label{sec:supuestos del mercado}
Para cada experimento, se consideraron los siguientes supuestos:

\begin{itemize}
\item El precio de ejecución en el día $t$ será el promedio entre el precio máximo y el precio mínimo de ese día.

\item Dada una señal de compra o venta en el día $t$, la acción se ejecuta en el día $t+1$ utilizando el precio de ejecución.

\item El costo de cada transacción se fija en $0.25\%$. Así para cada compra con precio de ejecución $P_{C}$ terminamos pagando un total de 
$$P_{C}(1 + 0.0025)$$
por cada acción comprada.

De la misma forma, por cada acción que vendemos a un precio $P_{V}$, recibimos 
$$P_{V}(1 - 0.0025)$$

\item En cada señal de compra, compramos tantas acciones podamos comprar con nuestro capital disponible (considerando el costo de transacción).

\item Para cada señal de venta, vendemos todas las acciones que poseamos.

\item Sólo podemos vender cuando tenemos acciones en nuestra posesión, es decir, no se permiten ventas en corto.

\item El límite superior para las señales de venta, $LS$, se fija con un valor de $0.035$ o $3.5\%$ relativo al último precio de compra.

\item El límite inferior para las señales de venta, $LI$, se fija con un valor de $-0.03$ o $-3.0\%$ relativo al último precio de compra. 
\end{itemize}

%=============== Resultados ================= %
\chapter{Resultados experimentales}
\label{capitulo:resultados experimentales}

%=============== Conclusión ================= %
\chapter{Conclusiones y trabajo futuro}
\label{capitulo:conclusiones}

\section{Conclusiones}
\label{seccion:conclusiones}

\section{Trabajo futuro}
\label{seccion:trabajo futuro}


\bibliography{references}
\addcontentsline{toc}{chapter}{Referencias}
\bibliographystyle{plain}







\end{document}
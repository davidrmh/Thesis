\documentclass[12pt]{report}
\usepackage[utf8]{inputenc}
\usepackage[none]{hyphenat} %Para evitar el corte de palabras
\usepackage{amsmath}
\usepackage{amsthm} %Para definir ambientes con \newtheorem
\usepackage{amsfonts}
\usepackage{amssymb}
\usepackage{makeidx}
\usepackage{graphicx}
\usepackage[square,sort,comma,numbers]{natbib}
\usepackage{url}
\usepackage{pdfpages} %para incluir la portada

\bibliographystyle{elsarticle-harv}

\title{Aprendizaje de reglas para operar en el mercado accionario}
%\publishers{Centro de Investigación en Computación, Instituto Politécnico Nacional}
\date{}
\author{David Ricardo Montalván Hernández}
	
%=========Define los ambientes a utilizar =======%
%Define estilo para dar un salto de línea en el encabezado
%del 'teorema'
\newtheoremstyle{break}
{2ex} %above space
{2ex} %below space
{} %Body font \itshape)
{} %indent amount
{\bfseries} %head font
{:} %post head puncuation
{\newline} %post head space
{}

\theoremstyle{break}
%Definición
\newtheorem{definicion}{Definición}[chapter]

%Teorema
\newtheorem{teorema}{Teorema}[chapter]

%Notas importantes
\newtheorem{nota}{Nota}[chapter]

%Ejemplos
\newtheorem{ejemplo}{Ejemplo}[chapter]

%Algoritmo (Utiliza el ambiente tabbing)
\theoremstyle{break}
\newtheorem{algoritmo}{Algoritmo}[chapter]
%=================================================%

%====================Macros=======================%
%\newcommand{\buyhold}{\textit{compra y espera}}
%=================================================%

\begin{document}
\sloppy %Para justificar correctamente (tiene que ver con \usepackage[none]{hyphenat})
\includepdf{portada/portada.pdf}
%\maketitle
\pagenumbering{Roman} %numeración romana con mayúsculas

\chapter*{Dedicatoria}
A mis padres y amigos.

\chapter*{Agradecimientos}
Agradezco a mi director de tesis el doctor Salvador Godoy Calderón por su apoyo en la realización de este trabajo.\newline
Agradezco también a cada uno de los miembros de mi comité tutorial por sus valiosas observaciones.

\renewcommand{\contentsname}{Contenido}
\tableofcontents
\renewcommand{\listfigurename}{Lista de figuras}
\listoffigures
\renewcommand{\listtablename}{Lista de tablas}
\renewcommand\tablename{Tabla}
\renewcommand{\bibname}{Referencias}
\renewcommand{\figurename}{Figura}
\renewcommand{\chaptername}{Capítulo}
\listoftables

\chapter*{Resumen}
El incremento en el poder de cómputo, la digitalización de los mercados financieros y la oportunidad de obtener grandes ganancias, son algunos de los factores que han motivado la investigación y desarrollo de algoritmos computacionales enfocados a guiar la toma de decisiones de inversión. En particular, se busca que estos algoritmos sean capaces de determinar los momentos adecuados para realizar compras o ventas de un activo financiero.

En este trabajo se propone una metodología para aprender, de manera automática, un conjunto de reglas de la forma \textit{si...entonces}, las cuales nos permitirán generar un conjunto de estrategias de inversión. Estas estrategias serán probadas utilizando datos del mercado accionario mexicano y estadounidense.

Además, desde el punto de vista de la teoría económica, este trabajo investiga la hipótesis del mercado eficiente en su versión débil.

\chapter*{Abstract}
The increase in the computing power, the digitalization of financial markets and the opportunity of making big profits, are among the factors that have influenced the research and development of computational algorithms whose objective is to guide the investment decisions process. Concretely, these algorithms should be able to determine when is the right moment to buy or sell a financial asset.

In this work, a methodology for automatically learning a set of \textit{If...then} rules that will allow us to generate a set of trading strategies is proposed. These strategies will be tested using data from the Mexican and American stock market.

Added to this, from the point of view of economic theory, this work investigates the weak version of the efficient market hypothesis.

%\chapter*{The modeler's hippocratic oath}
%\begin{itemize}
%\item I will remember that I didn’t make the world, and it doesn’t satisfy my equations.
%\item Though I will use models boldly to estimate value, I will not be overly impressed by mathematics.
%\item I will never sacrifice reality for elegance without explaining why I have done so.
%\item Nor will I give the people who use my model false comfort about its accuracy. Instead, I will make explicit its assumptions and oversights.
%\item I understand that my work may have enormous effects on society and the economy, many of them beyond my comprehension.
%\end{itemize}

%\begin{center}
%\textit{Paul Wilmott y Emmanuel Derman.}
%\end{center}

%=============== INTRODUCCIÓN ================= %
\chapter[Capítulo \thechapter: Introducción]{Introducción}
\pagenumbering{arabic} %Numeración árabe
\label{capitulo:introduccion}
%Motivación y ¿qué es lo que se busca con este trabajo?
%Objetivo general y objetivos particulares.
%Estructura del trabajo.
Un mercado es un entorno (no necesariamente físico) en el cual se reúnen compradores y vendedores con el fin de intercambiar bienes y/o servicios. En particular, en el mercado accionario se intercambian acciones de compañías (ver definición \ref{definicion:instrumento-financiero}). Hablando someramente, los compradores de acciones están motivados por su pronóstico de que dichas acciones incrementarán su valor en un futuro próximo; de forma que compran en el presente y esperan vender en el futuro
con el fin de obtener una ganancia. Por otra parte, los vendedores de acciones están motivados por su pronóstico de que el valor de dichas acciones disminuirá a corto plazo, por lo que prefieren vender en el presente, antes de que el precio caiga.

Dado lo anterior, contar con herramientas que ayuden a determinar los momentos correctos de compra y venta es claramente de gran utilidad. A pesar de que la idea básica es comprar barato y vender caro, la incertidumbre y complejidad de los mercados financieros, así como el incremento en el poder de cómputo y la digitalización de los mercados, han dado lugar al uso de herramientas computacionales enfocadas a guiar la toma de decisiones, concretamente, técnicas relacionadas al aprendizaje de máquina han ganado notoriedad \cite{AdvancesMLFinance}.

\section{Objetivo general}
\label{seccion:objetivo general}
El objetivo general de este trabajo es proponer una metodología para aprender, de manera automática, un conjunto de reglas. Estas reglas se utilizan para dirigir estrategias de inversión, las cuales tendrán una ganancia superior a la ganancia generada por la estrategia de \textit{compra y espera}, estrategia que es usada comúnmente como un marco de referencia (ver sección \ref{seccion:mercado eficiente}).

Además, la metodología propuesta busca obtener un modelo interpretable, contrastando así con la tendencia actual que se caracteriza por el uso de las llamadas \textit{cajas negras}, por ejemplo redes neuronales (ver sección \ref{capitulo:antecedentes}).

\section{Objetivos particulares}
\label{seccion:Objetivos particulares}
Los objetivos particulares de este trabajo son:
\begin{itemize}

\item Seleccionar la fuente de datos.

\item Proponer una metodología para etiquetar los datos y así poder aplicar algoritmos de aprendizaje supervisado.

\item Seleccionar la forma de separar los conjuntos de entrenamiento y prueba.

\item Proponer una metodología para aplicar un aprendizaje incremental. Esta metodología permitirá que el modelo aprenda de los errores cometidos.

\item Validar los resultados obtenidos, comparando con el exceso de ganancia sobre la estrategia compra y espera.
\end{itemize}

\section{Estructura del trabajo}
\label{seccion:estructura del trabajo}
La estructura de este trabajo es la siguiente:
En la sección \ref{capitulo:antecedentes}, se dan los antecedentes del problema y se exploran las distintas metodologías que han sido empleadas con el fin de generar estrategias de inversión. Se hace un análisis comparativo entre los enfoques utilizados así como los tipos de mercado analizados.

En la sección \ref{capitulo:marco teorico}, se da un panorama general sobre el funcionamiento de los mercados de índices accionarios, se explica la hipótesis del mercado eficiente así como sus posibles versiones y el tipo de información relacionada a cada una de ellas. Se explica la estrategia \textit{compra y espera} y la métrica de desempeño que se utiliza para evaluar las estrategias aprendidas. En la sección \ref{subseccion:indicadores tecnicos} se explican los atributos que se utilizan y en las sección \ref{seccion:algoritmos-aprendizaje-reglas} los algoritmos de inducción de reglas utilizados.

En la sección \ref{capitulo:solucion propuesta}, se explican los datos utilizados y la metodología propuesta, se establece además, la metodología de etiquetado y la forma de separar los conjuntos de entrenamiento y prueba. Finalmente se explica la forma en que se realiza el aprendizaje incremental propuesto, así como los supuestos bajo los cuales se trabajó.

La sección \ref{capitulo:resultados experimentales} contiene los resultados experimentales de cada uno de los experimentos.

Para finalizar, la sección \ref{capitulo:conclusiones} trata las conclusiones obtenidas así como las posibles direcciones para extender este trabajo.

\section{Contribuciones}
\label{seccion:contribuciones}
Una de las principales contribuciones de este trabajo, es el análisis del comportamiento del mercado accionario mexicano desde un enfoque basado en reglas. Después de analizar exhaustivamente el estado del arte, no se encontró evidencia alguna de otras investigaciones que utilicen el enfoque de reglas para analizar el mercado accionario mexicano, por lo que se asume que este trabajo es pionero en ese aspecto.

Otra contribución de este trabajo, radica en el método propuesto para etiquetado de las series de tiempo analizadas. En esta metodología se hace uso de un algoritmo evolutivo, que también es una propuesta original, para etiquetar cada observación y proporcionar una referencia a los posteriores algoritmos de aprendizaje.

%=============== Antecedentes ================= %
\chapter[Capítulo \thechapter: Revisión del estado del arte]{Revisión del estado del arte}
\label{capitulo:antecedentes}
%orden cronológico
%Tabla con referencia, mercado, metodologia
%Tabla con referencia, vence bh, considera costos de transacción

Al analizar el estado del arte, se puede observar que las heurísticas evolutivas han jugado un papel esencial para obtener un conjunto de reglas cuyo fin es guiar las decisiones de compra y venta de acciones \cite{Allen1999}, \cite{Canelas2012-gecco}, \cite{Canelas2013-gecco}, \cite{Canelas2013-journal}, \cite{Leitao2016}, \cite{Potvin2004}, \cite{Kampouridis2017}. Además, los atributos utilizados en la mayoría de los trabajos, son un conjunto de indicadores técnicos (ver sección \ref{subseccion:indicadores tecnicos}).

Las tablas \ref{tabla:referencia-mercado-tecnica} y \ref{tabla:referencia-venceBH} muestran un resumen de las referencias recolectadas para este trabajo, como se puede observar existen resultados mixtos en cuanto a obtener mejores ganancias que la estrategia \textit{compra y espera} (ver sección \ref{seccion:buy and hold}), la cual es comúnmente  utilizada como la estrategia de referencia. Por otra parte, se observa que las técnicas de aprendizaje utilizadas, pertenecen, principalmente, al paradigma subsimbólico (ver sección \ref{seccion:algoritmos-aprendizaje-reglas}). En este trabajo se propone una metodología simbólica para aprender el conjunto de reglas de interés (ver sección \ref{subseccion:algoritmo aq}).

\section{Programación genética}
De acuerdo a \cite{simon2013evolutionary}, la programación genética es un intento de generalizar los algoritmos evolutivos, buscando no solamente encontrar la solución óptima de un problema, si no que, además, se busca la estructura óptima para este. En otras palabras, la programación genética evoluciona programas que computan soluciones óptimas.

Uno de los primeros artículos en donde se utilizan técnicas relacionadas al aprendizaje de reglas es \cite{Allen1999}, en este trabajo se utiliza una metodología basada en programación genética, obteniéndose de esta manera un conjunto de reglas \textit{si...entonces}, para operar en el mercado. Los autores analizan el mercado accionario de Estados Unidos utilizando la información del índice accionario S\&P 500 (ver sección \ref{seccion:indices accionarios}) y comparándose contra la estrategia \textit{compra y espera} (ver sección \ref{seccion:buy and hold}) obteniendo resultados poco favorables. 

En la figura \ref{imagen:progracion genetica}, se ilustra un ejemplo de regla aprendida utilizando programación genética. En este caso, si el precio promedio de $50$ días es menor al precio actual, entonces se ejecuta una señal de compra, en el caso contrario se ejecuta una señal de venta. Es importante señalar que en \cite{Allen1999} no se consideran señales de espera, es decir, señales que indican que no se hace una compra ni venta.

\begin{figure}[ht]
\centering
\scalebox{0.6}{\includegraphics[width=.8\linewidth]{imagenes/programacion_genetica.png}}
\caption{\label{imagen:progracion genetica} Ejemplo de una regla para operar en el mercado, obtenida a través de programación genética}
\end{figure}

Los autores en \cite{Potvin2004}, retoman el trabajo de \cite{Allen1999} modificando la forma en que se crean los árboles y considerando $14$ compañías canadienses que forman parte del índice accionario TSE $300$. A pesar de que se vence a la estrategia \textit{compra y espera} en $9$ de las $14$ compañías, no se consideran costos de transacción\footnote{Se entiende por costos de transacción, a las comisiones que son cobradas por realizar una compra o una venta.}, limitando con esto la aplicación práctica de la metodología. 

Los trabajos \cite{Allen1999} y \cite{Potvin2004}, se retoman en \cite{Lohpetch2010}, en donde no sólo se usan precios diarios, si no que también se utilizan precios semanales y mensuales, además, en el proceso evolutivo que crea los árboles, se utiliza una función de aptitud que penaliza la complejidad de estos, evitando así un sobre ajuste en los datos de entrenamiento. En este trabajo se logra vencer a la estrategia \textit{compra y espera} aún considerando costos de transacción.

\section{Empate de patrones}
En \cite{Leigh2002}, se utiliza un enfoque basado en empate de patrones (en inglés, template match). Esta metodología se basa en detectar patrones en las gráficas de los precios, concretamente, se busca el patrón llamado \textit{bull flag}, el cual es considerado como un patrón que indica el inicio de una tendencia a la alza en los precios (ver figura \ref{imagen:bull flag}). Una vez detectado dicho patrón, se realiza la compra de acciones, las cuales son vendidas $100$ días después de su compra. Este artículo utiliza la estrategia \textit{compra y espera} como referencia, obteniendo mayores ganancias que ésta, sin embargo, no se consideran costos de transacción, por lo que la metodología se vuelve poco realista. El mercado analizado es el mercado de Estados Unidos a través del índice accionario NYSE Composite.

\begin{figure}[ht]
\centering
\scalebox{0.6}{\includegraphics[width=.8\linewidth]{imagenes/bull-flag.png} }
\caption{\label{imagen:bull flag} Patrón \textit{bull flag} que señala una futura tendencia a la alza en los precios.}
\end{figure}

En \cite{Parracho2010}, se retoma el trabajo de \cite{Leigh2002} introduciendo un algoritmo genético para realizar un proceso de optimización sobre parámetros que antes se consideraban fijos. Se obtienen resultados favorables pero sin considerar costos de transacción, el patrón \textit{bull flag} es nuevamente el único patrón considerado.

\section{PIP's y SAX}
En la  serie de artículos \cite{Canelas2012-gecco}, \cite{Canelas2013-gecco}, \cite{Canelas2013-journal}, \cite{Leitao2016} se desarrolla una metodología en la cual se busca reducir la dimensión de la serie de tiempo a través de la detección de los \textit{puntos perceptualmente importantes} (perceptually important points, PIPs). Una vez identificados estos puntos, se calcula la variación porcentual entre ellos y de acuerdo a esta cantidad la serie de tiempo es convertida en una cadena de símbolos (symbolic aggregate approximation, SAX). Estos símbolos son comparados con un patrón base, el cual se obtiene a través de una heurística evolutiva. De acuerdo a esta comparación, las señales de compra son generadas. Para realizar una venta, se consideran tres criterios:
\begin{enumerate}
\item Tiempo: Se vende si han transcurrido un determinado número de días después de la compra.
\item Precio: Se vende si el precio a alcanzado un nivel objetivo.
\item Patrón: Se vende si se detecta un patrón de venta (de la misma forma que el patrón de compra, el patrón de venta se obtiene a través de una heurística evolutiva).
\end{enumerate}
 Se vence a la estrategia \textit{compra y espera} pero no se consideran costos de transacción. La figura \ref{imagen:pip sax} ilustra esta metodología.

\begin{figure}[ht]
\centering
\scalebox{1.0}{\includegraphics[width=.8\linewidth]{imagenes/pip-sax.png} }
\caption{\label{imagen:pip sax} Ilustración de la metodología PIP / SAX.}
\end{figure}


\section{Otras metodologías}
Una línea totalmente distinta a los artículos anteriores es el trabajo de \cite{Kaucic2010}. En este, no se busca obtener un conjunto de reglas \textit{si...entonces}, si no que busca encontrar, a través de un aprendizaje evolutivo, la mejor combinación de indicadores técnicos (ver sección \ref{seccion:analisisTecnico}) con la cual se busca determinar los momentos adecuados para comprar o vender. Únicamente se logra vencer a la estrategia \textit{compra y espera} en mercados a la baja o estables (considerando costos de transacción). Se analiza el mercado accionario de Estados Unidos.

En \cite{Teixeira2010}, se logra vencer a la estrategia \textit{compra y espera} (considerando costos de transacción) utilizando una metodología basada en el algoritmo \textit{k nearest neighbors} (K-NN). Se analizan $15$ series accionarias pertenecientes al mercado brasileño. Una propuesta interesante de este artículo, es el uso de límites que capturan el perfil de riesgo de los inversionistas\footnote{En finanzas estos límites son llamados stop loss y take profits} (ver sección \ref{seccion:limites ventas}).

En \cite{Hu2015-XCS}, se utiliza aprendizaje por reforzamiento junto con un algoritmo evolutivo para obtener un conjunto de reglas \textit{si..entonces}.\footnote{Extended Classifier System, XCS} Estas reglas son aplicadas para operar en el mercado de índices accionarios de la bolsa de Shangai, se vence a la estrategia \textit{compra y espera} aún considerando costos de transacción.

Dentro del paradigma conexionista, en \cite{Sezer2017}, se utiliza una red neuronal con el fin de obtener las decisiones de compra y venta. Como atributos se seleccionan un conjunto de indicadores técnicos. Se consideran costos de transacción y se obtienen resultados mixtos cuando las estrategias obtenidas se comparan con las estrategia \textit{compra y espera}. 

\section{Etiquetado de datos}
Para finalizar este capítulo, es importante resaltar que dentro de las metodologías que aplican un aprendizaje supervisado, no se puede observar una explicación clara sobre la forma en que se asignan etiquetas a los datos. En este trabajo se propone un etiquetado basado en una heurística evolutiva (ver sección \ref{seccion:proceso etiquetado}) la cual busca encontrar, una vez observada la historia del periodo bajo análisis, la mejor combinación de compras y ventas que habrían generado la mayor ganancia.
 
\begin{center}
\begin{table}[htbp]
\centering
\begin{tabular}{ccc}
\hline
\textbf{Referencia} & \textbf{Mercado} & \textbf{Técnica} \\
\hline
\cite{Allen1999} & Estados Unidos & Programación genética \\
\cite{Leigh2002} & Estados Unidos & Emparejamiento de patrones\\
\cite{Potvin2004} & Canadá & Programación genética\\
\cite{Parracho2010} & Estados Unidos & Emparejamiento de patrones\\
\cite{Kaucic2010} & Estados Unidos & Aprendizaje evolutivo\\
\cite{Lohpetch2010} & Estados Unidos & Programación genética\\
\cite{Teixeira2010} & Brasil & K-NN\\
\cite{Preen2010} & Estados Unidos & Aprendizaje evolutivo\\
\cite{Esfahanipour2011} & Irán & Programación genética\\
\cite{Canelas2012-gecco}, \cite{Canelas2013-gecco}, \cite{Canelas2013-journal}, \cite{Leitao2016} & Estados Unidos & PIP-SAX \\
\cite{Kuo2013} & Taiwan & Búsqueda tabú \\
\cite{Wang2014} & Estados Unidos & Enjambre de partículas \\
\cite{Hu2015-XCS} & Shangai & XCS \\
\cite{Huang2015} & Varios & Biagrupamiento \\
\cite{Kim2016} & Corea & Conjuntos rugosos \\
\cite{Kampouridis2017} & Tipo de cambio & Aprendizaje evolutivo\\
\cite{Sezer2017} & Estados Unidos & Redes neuronales\\
\cite{Alimoradi2018} & Irán & Q-learning\\
\hline
\end{tabular}
\caption{\label{tabla:referencia-mercado-tecnica} Comparativo del estado del arte por mercado y técnica utilizada}
\end{table}
\end{center}

\begin{center}
\begin{table}[htbp]
\centering
\begin{tabular}{ccc}
\hline
\textbf{Referencia} & \textbf{Vence \textit{compra y espera}} & \textbf{Considera costos de transacción} \\
\hline
\cite{Allen1999} & No & Si \\
\cite{Leigh2002} & Si & No\\
\cite{Potvin2004} & Si (periodos estables o a la baja)  & No\\
\cite{Parracho2010} & Si & No\\
\cite{Kaucic2010} & Si (periodos estables o a la baja) & Si\\
\cite{Lohpetch2010} & Si & Si\\
\cite{Teixeira2010} & Si & Si\\
\cite{Preen2010} & Si (mayoría de los casos) & Si\\
\cite{Esfahanipour2011} & Si & Si\\
\cite{Canelas2012-gecco}, \cite{Canelas2013-gecco}, \cite{Canelas2013-journal}, \cite{Leitao2016} & Si & No \\
\cite{Kuo2013} & Si & Si \\
\cite{Wang2014} & No hay comparación & Si \\
\cite{Hu2015-XCS} & Si & Si \\
\cite{Huang2015} & Si & No \\
\cite{Kim2016} & Si & No \\
\cite{Kampouridis2017} & Si & Si\\
\cite{Sezer2017} & Resultados mixtos & Si\\
\cite{Alimoradi2018} & Si & Si\\
\hline
\end{tabular}
\caption{\label{tabla:referencia-venceBH} Comparativo del estado del arte por desempeño respecto a la estrategia \textit{compra y espera}}
\end{table}
\end{center}


%=============== Marco teórico ================= %
\chapter[Capítulo \thechapter: Marco teórico]{Marco teórico}
\label{capitulo:marco teorico}

\section{Mercado de índices accionarios}
\label{seccion:indices accionarios}

Se inicia esta sección con algunas definiciones básicas \cite{CFA2019-market-org}.

\begin{definicion}[Mercado]
\label{definicion:mercado}
Un \textbf{mercado} es un entorno (no necesariamente físico) en donde dos partes se reúnen para realizar el intercambio de bienes o servicios.
\end{definicion}

\begin{definicion}[Instrumento financiero y acciones]
\label{definicion:instrumento-financiero}
Un \textbf{instrumento financiero} es un contrato monetario entre dos partes. Si este contrato da derecho a poseer una proporción de los activos y ganancias de la compañía, entonces recibe el nombre de \textbf{acción}.
\end{definicion}

Las empresas que requieren fondos para financiar sus proyectos, pueden recurrir a la emisión de acciones con el fin de obtener los recursos necesarios. Por otra parte, los inversionistas que proyectan un crecimiento futuro de la empresa (y por lo tanto un crecimiento en el precio de sus acciones), son los proveedores de dichos fondos, así pues, existe un intercambio de acciones por dinero.

\begin{definicion}[Mercado financiero]
\label{definicion:mercado-financiero}
Un \textbf{mercado financiero} es un mercado en el cual se intercambian instrumentos financieros, en particular, acciones de compañías.
\end{definicion}

Debido a la diversidad de los tipos de mercados financieros (accionario, deuda, tipo de cambio, derivados, etc.) así como al funcionamiento particular de cada uno de ellos, este trabajo  únicamente analiza el mercado de índices accionarios.

\begin{definicion}[Índice accionario]
\label{definicion:indice-accionario}
Un \textbf{índice accionario}, es un indicador que busca representar el comportamiento de un segmento específico del mercado financiero accionario, por ejemplo un sector o un área geográfica en particular \cite{CFA2019-market-index}.
\end{definicion}

Así pues, el propósito principal de este tipo de índices es representar la opinión colectiva que se tiene respecto a un conjunto de acciones.

En este trabajo consideraremos lo siguientes índices :

\begin{itemize}
\item Índice de precios y cotizaciones (S\&P/BMV IPC)
\item Índice Standard and Poors 500 (S\&P 500)
\end{itemize}

El índice de precios y cotizaciones es el principal indicador del mercado accionario mexicano. Este índice busca medir el desempeño de las acciones listadas en la Bolsa Mexicana de Valores utilizando una muestra con las 35 series accionarias de mayor tamaño y liquidez listadas en dicha bolsa\footnote{https://espanol.spindices.com/indices/equity/sp-bmv-ipc}.

Por otro lado, el índice Standard and Poors 500 es considerado como el mejor indicador para el mercado accionario de Estados Unidos. Este índice está compuesto por las 500 empresas de mayor capitalización, las cuales capturan alrededor del $80\%$ de la capitalización del mercado estadounidense.\footnote{https://us.spindices.com/indices/equity/sp-500}

\begin{nota} \label{nota:ETF}
En la práctica los inversionistas no pueden comprar los índices accionarios, en cambio, los instrumentos negociados reciben el nombre de Títulos Referenciados a Acciones (TRACS\footnote{En inglés Exchange Traded Funds o ETF}). Estos instrumentos buscan replicar el comportamiento del índice al que están relacionados.

El instrumento que busca replicar el índice de precios y cotizaciones recibe el nombre de iShares NAFTRAC (o simplemente NAFTRAC), mientras que para el índice Standard and Poors 500 el instrumento relacionado es el SPDR S\&P 500.
\end{nota}

\section{Mercado eficiente}
\label{seccion:mercado eficiente}
El concepto de mercado eficiente, hace referencia a un mercado que es capaz de capturar, de forma rápida y precisa, la información\footnote{Información relacionada a la economía en general.} disponible más reciente. Esta información es reflejada en los precios de los instrumentos financieros y como consecuencia, en un mercado eficiente, los precios son informativos, lo que permite lograr una colocación eficiente de los recursos (ver \cite{Fama1965} y \cite{CFA2019}).

\subsection{Versiones de eficiencia en el mercado}
\label{subseccion:versiones emh}
Eugene Fama describe tres formas de eficiencia de un mercado: débil, semifuerte y fuerte \cite{Fama1965}.

En la versión débil, los precios actuales reflejan toda la información histórica del mercado, esta información hace referencia a los precios y volúmenes negociados en el pasado. En consecuencia, bajo esta versión de eficiencia, no es posible utilizar precios y volúmenes históricos con el fin de predecir la tendencia futura de los precios (ver sección \ref{seccion:analisisTecnico}).

La versión semifuerte incluye la versión débil y añade que, además de reflejar la información histórica del mercado, los precios actuales reflejan toda la información pública. La información pública es aquella a la cual el público inversionista puede acceder, por ejemplo, noticias económicas o financieras, estados de resultados de las empresas, reportes anuales o trimestrales de las mismas, indicadores macroeconómicos etc. De acuerdo a esta versión, no es posible utilizar dicha información con el fin de predecir la tendencia futura de los precios.

Finalmente, la versión fuerte de la eficiencia de un mercado, nos dice que los precios actuales reflejan totalmente tanto la información pública como la información privada, es decir, aquella que solamente algunas personas poseen. Por lo tanto, bajo esta versión, incluso tomando en cuenta la información privada, no es posible predecir la tendencia futura de los precios.

De acuerdo a lo anterior, la eficiencia de un mercado debe ser interpretada relativa al tipo de información que los inversionistas utilizan. Además, como se señala en \cite{CFA2019}, la eficiencia cae en un espectro continuo, que varía a través del tiempo, regiones geográficas y tipo de mercado.

La figura \ref{imagen:versiones emh} ilustra las distintas versiones de la eficiencia de un mercado.

Este trabajo se enfoca únicamente en la versión débil de la eficiencia de un mercado.

\begin{figure}[ht]
\centering
\scalebox{0.8}{\includegraphics[width=.8\linewidth]{imagenes/versiones-emh.jpeg}}
\caption{\label{imagen:versiones emh} Versiones de la eficiencia de un mercado}
\end{figure}


\subsection{Estrategia \textit{compra y espera}}
\label{seccion:buy and hold}
Como se señala en la sección \ref{subseccion:versiones emh}, de acuerdo a la hipótesis del mercado eficiente, no es posible utilizar información histórica con el fin de predecir los precios del futuro. Lo mejor que se puede hacer es comprar un portafolio bien diversificado y mantenerlo por un periodo de tiempo predeterminado, esta estrategia recibe el nombre de estrategia \textit{compra y espera}.

De acuerdo a lo anterior se tiene que, en un mercado eficiente, no es posible obtener, de manera consistente y sistemática, ganancias superiores a aquellas generadas por la estrategia \textit{compra y espera}.

Dado un plazo de tiempo fijo, $\left[T_{I}, T_{F}\right]$, la ganancia de la estrategia \textit{compra y espera} está dada por la ecuación (\ref{eqn:ganancia BH})

\begin{equation} \label{eqn:ganancia BH}
G_{\left[T_{I}, T_{F}\right]} (BH) = \dfrac{P_{T_F} (1 - c) } { P_{T_I} (1 + c) } - 1
\end{equation}

en donde $P_{T_F}$ es el precio de ejecución al final del plazo, $P_{T_I}$ es el precio de ejecución al inicio de este y $c$ es el costo (porcentual) de cada transacción, es decir, la comisión que se debe de pagar por ejecutar cada transacción (ver sección \ref{sec:supuestos del mercado}). 

La figura \ref{imagen:buy hold alza} ejemplifica la estrategia \textit{compra y espera} en un periodo en el cual el mercado presenta una tendencia a la alza. Como se puede observar, vencer esta estrategia cuando se presenta este tipo de tendencia es todo un reto, esto se debe a que sólo se involucran dos operaciones, lo que implica un ahorro en costos de transacción.

\begin{figure}[ht]
\centering
\scalebox{0.8}{\includegraphics[width=1\linewidth]{imagenes/buy-hold-alza.jpeg}}
\caption{\label{imagen:buy hold alza} Estrategia \textit{compra y espera} en un mercado a la alza}
\end{figure}

\subsection{Exceso de ganancia sobre compra y espera}
\label{subseccion:exceso de ganancia}

Para comparar las estrategias propuestas en este trabajo contra la estrategia \textit{compra y espera}, se utiliza como métrica de desempeño el exceso de ganancia (porcentual).

Esta métrica compara la ganancia obtenida por una estrategia, $S$, con aquella obtenida por la estrategia \textit{compra y espera}.

\begin{definicion}[Exceso de ganancia]
\label{definicion:exceso de ganancia}
Para un plazo fijo, $\left[T_{I}, T_{F}\right]$, el exceso de ganancia de una estrategia $S$, sobre la estrategia \textit{compra y espera}, está dado por la ecuación (\ref{eqn:Exceso de ganancia})

\begin{equation} \label{eqn:Exceso de ganancia}
ExG_{\left[T_{I}, T_{F}\right]} = G_{\left[T_{I}, T_{F}\right]} (S) - G_{\left[T_{I}, T_{F}\right]} (BH)
\end{equation}

en donde la ganancia $G_{\left[T_{I}, T_{F}\right]} (BH)$, está dada por la ecuación (\ref{eqn:ganancia BH}) y la ganancia $G_{\left[T_{I}, T_{F}\right]} (S)$ es simplemente la variación porcentual entre el capital inicial disponible en el momento $T_{I}$ y el capital con el que se cuenta en el momento $T_{F}$, este capital se obtiene al ejecutar una serie de compras y ventas dentro del intervalo $\left[T_{I}, T_{F}\right]$.

\begin{equation} \label{eqn:Ganancia estrategia S}
G_{\left[T_{I}, T_{F}\right]}(S) = \dfrac{Capital_{T_F}}{Capital_{T_I} } - 1 
\end{equation}
\end{definicion}
Claramente buscamos que $ExG_{\left[T_{I}, T_{F}\right]}$ sea un número positivo.

\section{Análisis técnico}
\label{seccion:analisisTecnico}
Contrario a lo que establece la hipótesis del mercado eficiente (sección \ref{seccion:mercado eficiente}), el análisis técnico busca predecir la tendencia futura de los precios, utilizando la información histórica del mercado (precios y volumen) \cite{murphy1999technical}.

Los tres supuestos en los que se basa este tipo de análisis son los siguientes:

\begin{itemize}
\item La información de todos los factores que afectan a la oferta y a la demanda se encuentra reflejada en el precio de los instrumentos financieros. En consecuencia, basta centrarse en el estudio de los precios para poder analizar el mercado.

\item Los precios se mueven siguiendo tendencias (a la alza o a la baja). El objetivo del análisis técnico es detectar, lo más pronto posible, el inicio y el final de estas tendencias.

\item Los patrones en los precios tienden a repetirse.
\end{itemize}


\subsection{Indicadores técnicos}
\label{subseccion:indicadores tecnicos}
Para realizar el pronóstico de las tendencias, el análisis técnico se auxilia de los llamados indicadores técnicos. Estos indicadores, son transformaciones aplicadas a los precios y volúmenes históricos utilizando una ventana (deslizante) de tiempo determinada \cite{murphy1999technical}, \cite{technicalAnalysisKirkPatrick}, \cite{encycoplediaTechnicalIndicators}.

Los indicadores técnicos utilizados en este trabajo, son\footnote{Aunque en las fórmulas no se indica, cada indicador se calcula en cada período $t$ utilizando la información en el intervalo $\left[t-n + 1, t\right]$, es decir, la información contenida en una ventana deslizante de tamaño $n$. }:
\begin{itemize}
\item Oscilador aroon.

\item Relative strength index (RSI).

\item Money flow index (MFI).

\item Williams \%R.

\item Commodity channel index (CCI).
\end{itemize}


\subsubsection{Oscilador aroon}
\label{subsubseccion:Oscilador Aroon}
Este indicador mide el número de períodos, dentro de una ventana de tiempo de tamaño $n$, que han transcurrido desde el precio máximo y el precio mínimo más reciente. El cálculo de este indicador está dado en la ecuación (\ref{eqn:oscilador aroon})

\begin{equation} \label{eqn:oscilador aroon}
OsAroon = 100 \left( \dfrac{  n - T_H  } { n } - \dfrac{  n - T_L  } { n } \right)
\end{equation}

en donde $n$ es el número de períodos que comprende la ventana de tiempo, $T_H$ es el número de períodos transcurridos desde el último precio máximo registrado dentro de la ventana de tamaño $n$ y $T_L$ es el número de períodos transcurridos desde el último precio mínimo registrado dentro de la ventana de tamaño $n$.

La intuición de este indicador es la siguiente:

De acuerdo a la ecuación (\ref{eqn:oscilador aroon}), si $OsAroon$ es un número positivo, se tiene que $T_{H} < T_{L}$, es decir, ha transcurrido menos tiempo desde el último precio máximo (en la ventana de tiempo dada) que desde el último precio mínimo y por lo tanto los precios muestran una tendencia a la alza (debemos de prepararnos para comprar). En cambio cuando $OsAroon$ es un número negativo, se tiene que $T_{H} > T_{L}$, es decir, ha transcurrido menos tiempo desde el último precio mínimo que desde el último precio máximo y por lo tanto los precios muestran una tendencia a la baja (debemos de prepararnos para vender).

\subsubsection{Relative strength index (RSI)}
\label{subsubseccion:RSI}
Este indicador busca predecir la tendencia en los precios analizando el promedio de ganancias y pérdidas obtenidas para una ventana de tiempo de tamaño $n$. La fórmula para calcular el RSI está dada en la ecuación (\ref{eqn:RSI})

\begin{equation} \label{eqn:RSI}
RSI = 100 - \left( \frac{100}{1 + RS} \right)
\end{equation}

en donde:

\begin{equation} \label{eqn:RSI RS}
RS = \frac{Ganancia\,promedio\,en\,n\,periodos}{P\acute{e}rdida\,promedio\,en\,n\,periodos}
\end{equation}

Típicamente, de acuerdo a la literatura financiera (\cite{technicalAnalysisKirkPatrick}, \cite{encycoplediaTechnicalIndicators}) una señal de venta se genera cuando $RSI > 70$; por otra parte, una señal de compra se genera si $RSI < 30$.

\subsubsection{Money flow index (MFI)}
\label{subsubseccion:money flow index}
Este indicador es similar al RSI. Considera el número de períodos en los que se han presentado ganancias o pérdidas tomando en cuenta tanto los precios como el volumen del período (en comparación con el RSI que sólo considera los precios).

Para el cálculo del MFI se necesita primero calcular el precio promedio del período $t$, el cual está dado por la ecuación (\ref{eqn:precio tipico})

\begin{equation} \label{eqn:precio tipico}
Precio\, \, promedio_t = \dfrac{H_{t} + L_{t} + C_{t}}{3}
\end{equation}

en donde $H_{t}, L_t, C_t$ representan, respectivamente, el precio máximo, mínimo y de cierre en el período $t$.

Una vez calculado el precio promedio, se calcula el flujo de efectivo del período, como se muestra en la ecuación (\ref{eqn:flujo de efectivo})

\begin{equation} \label{eqn:flujo de efectivo}
MF_t = Precio\, \, promedio_t \times Volumen_t \times F_t
\end{equation}

en donde $F_t = 1$, si el precio promedio del período $t$ es mayor al precio promedio del período $t-1$, y $F_t = -1$ en el caso contrario.

Así, dada una ventana de tiempo de tamaño $n$, el MFI se calcula de acuerdo a la ecuación (\ref{eqn:MFI})

\begin{equation} \label{eqn:MFI}
MFI = 100 - \left( \frac{100}{1 + MFR} \right)
\end{equation}

en donde $MFR$ está dado por 

\begin{equation} \label{eqn:MFR}
MFR = \dfrac{\sum_{MF_t > 0} MF_t  }{\sum_{MF_t < 0} MF_t}
\end{equation}

con $t$ dentro del período de la ventana $n$.

En la práctica, es común considerar que una señal de venta se da cuando $MFI > 80$, mientras que una señal de compra se da cuando $MFI < 20$.

\subsubsection{Williams \%R}
\label{subsubseccion:Williams R}
Este indicador busca reflejar el precio de cierre relativo al precio máximo dentro de una ventana de tiempo dada.

Su cálculo está dado por la ecuación (\ref{eqn:williams R})

\begin{equation} \label{eqn:williams R}
Williams \%R_t = -100 \times \dfrac{HH - C_t}{HH - LL}
\end{equation}

en donde $HH$ es el máximo entre los precios máximos de cada período que comprende la ventana de tiempo, de manera similar $LL$ es el mínimo de los precios mínimos de cada período en la ventana de tiempo y $C_t$ es el precio de cierre en el tiempo $t$.

Se considera típicamente que si este indicador se encuentra entre $-20$ y $0$, es un momento apropiado para vender, mientras que si se encuentra dentro del rango de $-100$ a $-80$, es un momento adecuado para comprar.

\subsubsection{Commodity channel index (CCI)}
\label{subsubseccion:cci}
Este indicador mide la diferencia entre el precio promedio en el período $t$ (como se define en la ecuación (\ref{eqn:precio tipico})) y un promedio móvil de este mismo precio utilizando una ventana de tiempo de tamaño $n$. La ecuación (\ref{eqn:CCI}) muestra el cálculo del indicador CCI:

\begin{equation} \label{eqn:CCI}
CCI_t = \dfrac{ Prom_t - \overline{Prom_{t:n}} }{ 0.015 D_{t:n}}
\end{equation}

En donde $Prom_t$ está dado por la ecuación (\ref{eqn:precio tipico}), $\overline{Prom_{t:n}}$ es el promedio de $Prom_t$, considerando los períodos $t, t-1, \ldots, t-n +1$, finalmente el término $D$ es el promedio de las diferencias absolutas $\left|Prom_t - \overline{Prom_{t:n}}  \right|$.

Las señales de compra y venta del indicador CCI se basan en la hipótesis de que los precios presentan un comportamiento de reversión a la media, por lo tanto y en la práctica, valores mayores a $200$ en el CCI, son considerados como señales de venta (el precio tendrá una reversión a la media, por lo tanto disminuirá), mientras que valores menores a $-100$, son considerados señales de compra (el precio revierte a la media al aumentar su valor).

La tabla \ref{tabla:resumen reglas indicadores} muestra un resumen de las reglas de compra y venta para cada indicador técnico. Es importante señalar que los umbrales de cada decisión mostrada en la tabla, son determinados de forma empírica y no hay un sustento formal detrás de sus valores.
\begin{table}[h]
\centering
\begin{tabular}{ccc}
\hline
\textbf{Indicador} & \textbf{Regla de compra} & \textbf{Regla de  venta} \\
\hline
OsAroon & $OsAroon>0$ & $OsAroon<0$ \\
RSI & $RSI < 30$ & $RSI > 70$ \\
MFI & $MFI < 20$ & $MFI > 80$ \\
Williams \%R & $-100 \leq Williams \%R \leq -80 $ & $-20 \leq Williams \%R \leq 0$ \\
CCI & $CCI < -100$ & $CCI > 200$ \\

\hline
\end{tabular}
\caption{\label{tabla:resumen reglas indicadores} Reglas de compra y venta de los indicadores técnicos.}
\end{table}

%=============== AQ y CN2 ================= %
\section{Algo ritmos para aprendizaje de reglas}
\label{seccion:algoritmos-aprendizaje-reglas}
Esta sección sigue muy de cerca lo que se expone en \cite{foundations-rule}.

El aprendizaje de máquina es una área que estudia la forma de descubrir modelos, patrones y otras regularidades en un conjunto de datos. Esta área, se puede categorizar en tres grandes enfoques:

\begin{enumerate}
\item \textbf{Simbólico}: Se incluye el aprendizaje inductivo de descripciones simbólicas, tales como reglas, árboles de decisión o representaciones lógicas.

\item \textbf{Subsimbólico}: Aquí se encuentran métodos estadísticos y de reconocimiento de patrones, tales como: k-vecinos más cercanos, clasificadores bayesianos y máquinas de soporte vectorial.

\item \textbf{Conexionista}: En este enfoque se engloba el aprendizaje realizado por medio de redes neuronales.
\end{enumerate}

Este trabajo utiliza algoritmos que entran dentro del enfoque simbólico. Concretamente, los algoritmos utilizados inducen un conjunto de reglas de la forma \textit{si...entonces}. Cada regla tiene un antecedente formado por la conjunción de condiciones, cada una de estas condiciones involucra un atributo en particular. Como consecuente, la regla tiene una etiqueta de clase. El lenguaje utilizado para describir los datos es el lenguaje de atributo-valor.

En un enfoque basado en reglas, se dice que una regla \textbf{cubre} un ejemplo, si los valores de los atributos de este ejemplo cumplen las condiciones del antecedente de la regla. Se dice que una regla \textbf{cubre correctamente} un ejemplo, si además de cubrirlo, la clase del ejemplo corresponde a la clase en el consecuente de la regla.

Dado un conjunto de ejemplos (ejemplos positivos) que pertenecen a una misma clase (la clase objetivo o clase positiva) y un conjunto de ejemplos que no pertenecen a ella (ejemplos negativos), los algoritmos de inducción de reglas obtienen un conjunto de reglas (este conjunto es llamado hipótesis para la clase positiva) con las cuales se busca cubrir correctamente los ejemplos positivos, se dice entonces que estos algoritmos aprenden el concepto de la clase objetivo.

\begin{definicion}[Conjunto de reglas completo]
\label{definicion:reglas-completas}
Si un conjunto de reglas (hipótesis) para una clase objetivo (clase positiva) cubre correctamente todos los ejemplos positivos, entonces se dice que este conjunto de reglas es \textbf{completo}.
\end{definicion}

\begin{definicion}[Conjunto de reglas consistente]
\label{definicion:reglas-consistentes}
Si un conjunto de reglas (hipótesis) para una clase objetivo (clase positiva) no cubre ejemplo negativo alguno, se dice que es un conjunto de reglas \textbf{consistente}.
\end{definicion}

Finalmente, se distinguen entre dos tipos de tareas de aprendizaje:

\begin{enumerate}
\item \textbf{Aprendizaje supervisado}: Los ejemplos de entrenamiento están clasificados de acuerdo a la clase a la que pertenecen.

\item \textbf{No supervisado}: Los ejemplos de entrenamiento no se encuentran clasificados.
\end{enumerate}

A pesar de que los conjuntos de datos utilizados en este trabajo no se encuentran etiquetados, en la sección \ref{seccion:proceso etiquetado} se detalla una metodología para asignar una etiqueta de clase a cada observación, lo que permite utilizar técnicas de aprendizaje supervisado.

%\subsection{Algoritmos AQ y CN2}
%Falta alguna división
%Marco sobre algoritmos de aprendizaje automático
%\label{seccion:algoritmos aq cn2}

\subsection{Algoritmo AQ}
\label{subseccion:algoritmo aq}
El algoritmo AQ es un algoritmo de aprendizaje supervisado que induce un conjunto de reglas del tipo \textit{si...entonces} (ver \cite{AQCervone2010}, \cite{AQMichalski1991} y \cite{AQWojtusiak2012} para un estudio detallado). 

Dado un conjunto (descrito en el lenguaje atributo-valor), $P$, con $n$ observaciones, $P_1, P_2, \ldots, P_n$ (ejemplos de la clase positiva)  y un conjunto, $N$, con $m$ observaciones, $Q_1, Q_2, \ldots, Q_m$, (ejemplos de la clase negativa) el algoritmo AQ encuentra un conjunto de reglas que son completas y consistentes.
Las reglas tienen la forma dada en la expresión (\ref{eqn:reglas AQ})

\begin{equation} \label{eqn:reglas AQ}
Antecedente \rightarrow Consecuente
\end{equation}
en donde $Antecedente$ es una conjunción de selectores (esta conjunción recibe el nombre de complejo) y cada selector es de la forma

\begin{equation} \label{eqn:condicion AQ}
\left[Atributo\,\, OP\,\, Valores \right]
\end{equation}
en donde el término $OP$ depende del tipo de atributo que se está utilizando, por ejemplo, para un atributo con valores continuos tenemos que $OP \in \{>, \geq, <, \leq\}$.
La parte de el $Consecuente$ en (\ref{eqn:reglas AQ}), está formada por una sola condición, la etiqueta de la clase de la regla, por ejemplo la clase comprar.

El algoritmo AQ inicia su proceso de inducción seleccionando un ejemplo (la semilla), $S$, de la clase positiva, el cual es generalizado creando todos los complejos que cubren $S$ y que son consistentes, es decir, que no cubren ejemplo alguno de la clase negativa. Este conjunto de complejos recibe el nombre de estrella acotada y se denota como $G(S,N)$. El mejor complejo en la estrella es seleccionado de acuerdo a un criterio previamente establecido, en este trabajo, el criterio utilizado es maximizar el número de ejemplos positivos cubiertos. Este proceso es repetido hasta tener una disyunción de complejos (llamada cobertura) que es completa y consistente. El algoritmo \ref{algo:AQ} muestra un pseudocódigo para el algoritmo AQ.

\begin{algoritmo}[Algoritmo AQ]
\begin{tabbing}
\\$P$ el conjunto de ejemplos positivos de la clase positiva. 
\\$N$ el conjunto de ejemplos negativos.\\
1. \=$Cobertura\leftarrow \emptyset.$ \\
2. Mientras $P \neq \emptyset$:\\
 \>3. Elige semilla $S$ en $P.$\\
 \>4. Genera estrella $G(S,N).$\\
 \>5. Selecciona el mejor complejo, $c$, en $G(S,N).$\\
 \>6. Agrega $c$ a $Cobertura.$\\
 \>7. Elimina de $P$ los ejemplos cubiertos por $c.$\\
\=8. Regresa $Cobertura.$
\end{tabbing}
\label{algo:AQ}
\end{algoritmo}

\begin{algoritmo}[Genera estrella]
\begin{tabbing}
\\Sea $S$ la semilla correspondiente a un ejemplo de la clase positiva.
\\Sea $N$ el conjunto de ejemplos negativos.\\
1. \=$G(S,N)\leftarrow \emptyset$ (el complejo vacío cubre todas las observaciones).\\
2. Mientras $G(S,N)$ cubra algún ejemplo de $N$:\\
 \>3. Elige un ejemplo negativo $E_{neg}$ cubierto por $G(S,N).$\\
 \>4. \= Especializa los complejos en $G(S,N)$ con el fin de excluir $E_{neg}.$\\
 \>\>4.1 Sea $EX$ el conjunto de selectores que cubren $S$ pero no $E_{neg}.$ \\
 \>\>4.2 $G(S,N)=\{x \wedge y \vert x \in G(S,N), y \in EX \}.$\\
\=6. Regresa $G(S,N).$
\end{tabbing}
\label{algo:AQ genera estrella}
\end{algoritmo}

\subsection{Algoritmo CN2}
\label{subseccion:algoritmo cn2}
En el algoritmo \ref{algo:AQ genera estrella}, al momento de generar la estrella, el algoritmo AQ sólo considera especializaciones que excluyen un ejemplo negativo en particular, mientras que, al mismo tiempo, se busca cubrir la semilla. Como se señala en \cite{CN2-Clark1989}, este tipo de búsqueda se limita al espacio de complejos que son consistentes con el conjunto de entrenamiento, limitando así la capacidad de generalización del algoritmo.

El algoritmo CN2 remueve esta dependencia en ejemplos específicos y extiende el espacio de búsqueda al examinar todas las especializaciones posibles de un complejo.

Para encontrar el mejor complejo, el algoritmo CN2 utiliza dos heurísticas para guiar el proceso de creación de la estrella. La primera de ellas utiliza la entropía para medir la calidad de los complejos (entre menor entropía mejor es el complejo), así, dado un conjunto de ejemplos, $E$, cubiertos por el complejo $c$, si $p_i$ denota la probabilidad de la clase $i$ dentro de los ejemplos que pertenecen a $E$, entonces la entropía del complejo $c$, está dada por:

\begin{equation} \label{eqn:entropia complejo}
entrop(c) = - \sum_{i} p_i log_{2}(p_{i})
\end{equation}

La segunda heurística se utiliza para detectar complejos que son estadísticamente significativos. Dado un conjunto de ejemplos, $E$, cubiertos por un complejo $c$, esta heurística compara la distribución observada de las clases entre los ejemplos en $E$ y la distribución esperada de las clases en una muestra del mismo tamaño que $E$.

De esta manera, si $f_i$ es la frecuencia observada para la clase $i$ dentro de los ejemplos en $E$, $e_{i}$ es la frecuencia esperada de la clase $i$ para una muestra de tamaño $M=\vert E \vert$, es decir $e_i = M \times Prob\left[ Clase = i \right]$ y $k$ es el número de clases, entonces la significancia del complejo $c$, se calcula con la estadística\footnote{Esta estadística tiene una distribución $\chi^2$ con $k-1$ grados de libertad }  dada por la expresión (\ref{eqn:cn2 estadistica sig})

\begin{equation} \label{eqn:cn2 estadistica sig}
2 \sum_{i=1}^{k} f_{i} log\left(\frac{f_{i}} {e_{i}} \right)
\end{equation}

\begin{algoritmo}[Algoritmo CN2]
\begin{tabbing}
\\Sea $D$ el conjunto de datos etiquetados.
\\Sea $MC$ el mejor complejo.\\
1. \=$REGLAS$ $\leftarrow \emptyset.$ \\
2. Mientras $D \neq \emptyset$ o $MC \neq \emptyset$:\\
 \>3. Crea estrella y selecciona el mejor complejo $MC.$\\
 \>4. Sea $EC$ el conjunto de ejemplos cubiertos por $MC.$ \\
 \>5. Elimina $EC$ de $D.$\\
 \>6. Sea $CLA$ la clase más común en $EC.$\\
 \>7. Agrega la reglas "Si $MC$ entonces $CLA$" a $REGLAS.$\\
8. Regresa $REGLAS.$
\end{tabbing}
\label{algo:CN2}
\end{algoritmo}

%=============== Solución propuesta ================= %
\chapter[Capítulo \thechapter: Solución propuesta]{Solución propuesta}
\label{capitulo:solucion propuesta}

\section{Separación de datos}
\label{seccion:separacion de datos}
En este trabajo se utilizan precios diarios\footnote{Obtenidos de Yahoo Finance. https://finance.yahoo.com} de los instrumentos NAFTRAC y SPDR S\&P 500 (ver sección \ref{seccion:indices accionarios}).

Para el NAFTRAC, la información inicia el día 15 de febrero de 2013 y termina el 21 de enero de 2019. Para el SPDR S\&P 500 el período de tiempo abarcado es del 2 de enero de 2008 hasta  el 3 de marzo de 2019. Se tienen períodos de tiempo distintos debido a la calidad en los datos, en el caso del mercado mexicano se tienen un gran número de datos no disponibles para ciertos meses de los años $2008$ y $2012$.

La tabla \ref{tabla:Ejemplo datos diarios NAFTRAC} muestra la estructura de los datos para el NAFTRAC, es importante señalar que para ambos conjuntos de datos no se incluyen fines de semana o días festivos (relativos a cada país, por ejemplo, la información del 15 de septiembre no se incluye en el conjunto de datos del NAFTRAC, pero sí en el conjunto de datos del SPDR S\&P).

\begin{table}[h]
\centering
\begin{tabular}{ccccccc}
\hline
\textbf{Fecha} & \textbf{Apertura} & \textbf{Máximo} & \textbf{Mínimo} & \textbf{Cierre  ajustado} &  \textbf{Volumen} \\
\hline
2013/02/15 & 44.09 & 44.24 & 43.86  & 41.61 & 77,614,608\\
2013/02/18 & 44.38 & 44.38 & 44.04  & 41.60 & 6,457,428\\
2013/02/19 & 44.34 & 44.77 & 44.29  & 42.05 & 68,042,072\\
\vdots & \vdots & \vdots & \vdots  & \vdots & \vdots \\
\hline
\end{tabular}
\caption{\label{tabla:Ejemplo datos diarios NAFTRAC} Muestra de los datos diarios del NAFTRAC}
\end{table}

Para crear los conjuntos de entrenamiento y prueba, los datos se dividen utilizando una ventana deslizante de 90 días. Este criterio está fundamentado en el hecho de que, tanto en México como en Estados Unidos, las empresas con acciones listadas en alguna bolsa de valores, están obligadas a reportar los resultados de sus operaciones de manera trimestral; así pues, se busca capturar la reacción de los inversionistas ante la publicación de esta información.

La tabla \ref{tabla:data split SP500} muestra un ejemplo de la separación de los datos del SPDR S\&P 500.
\begin{table}[h]
\centering
\begin{tabular}{cc}
\hline
\textbf{Entrenamiento} & \textbf{Prueba} \\
\hline
2008/01/02 - 2008/05/09 & 2008/05/12 - 2008/09/17 \\
2008/05/12 - 2008/09/17 & 2008/09/18 - 2009/01/27 \\
2008/09/18 - 2009/01/27 & 2009/01/28 - 2009/06/05 \\
\vdots & \vdots \\
\hline
\end{tabular}
\caption{\label{tabla:data split SP500} Separación de datos}
\end{table}

Con esta forma de separar los datos, un conjunto de prueba se convierte en un nuevo conjunto de entrenamiento. De esta manera, se ajustan los modelos utilizando la información más reciente.

Utilizando esta metodología, se obtienen 15 conjuntos de entrenamiento y 15 conjuntos de prueba con la información del NAFTRAC. Por otra parte, con los datos del SPDR S\&P 500 se obtienen 30 conjuntos de entrenamiento y 30 conjuntos de prueba.

\section{Proceso de etiquetado}
\label{seccion:proceso etiquetado}
Ya que los datos de entrenamiento no están etiquetados, se utiliza un algoritmo evolutivo para obtener un conjunto de etiquetas que permita entrenar cada modelo. La idea detrás del etiquetado se basa en, una vez conocida toda la historia del período de entrenamiento, encontrar las combinaciones de compras y ventas que generen el mayor exceso de ganancia.

Las etiquetas utilizadas son:
\begin{itemize}
\item Venta, representada con $-1$.

\item Espera, representada con $0$.

\item Compra, representada con $1$ .
\end{itemize}

Este algoritmo evolutivo pertenece a la clase de algoritmos para la estimación de la distribución (EDA por sus siglas en inglés \cite{simon2013evolutionary}). Cada individuo en la población representa una estrategia de inversión para un periodo de entrenamiento dado. Esta estrategia se representa a través de un vector $\mathbf{x} = (a_1, a_2, \ldots, a_t)$ cuya componente $a_i$ representan la acción a tomar en el período $i$, $a_i \in \{-1,0,1\}.$ La función de aptitud utilizada es el exceso de ganancia sobre la estrategia \textit{compra y espera} (ver sección \ref{subseccion:exceso de ganancia}). Así, cada vector $\mathbf{x} = (a_1, a_2, \ldots, a_t)$, tiene asociado su respectivo exceso de ganancia de acuerdo a las compras y ventas que determinan sus componentes $a_i$; se busca aquel vector, $\mathbf{x}$, con mayor exceso de ganancia entre la población, este vector representa la mejor combinación de compras y ventas para el conjunto de datos dado.

Para generar las poblaciones de cada generación, este etiquetado utiliza lo que se conocen como estadísticas de primer orden, en particular, se utiliza la probabilidad de que cada componente $i$ tome un valor dentro del conjunto $\{-1, 0, 1\}$. A continuación se muestra un ejemplo.

\begin{ejemplo}
Para un período de 5 días, dados los mejores $5$ individuos de la generación $n$

$$
\begin{array}{ccc}
\mathbf{x}_1 & = & (1, 0, 0, -1, 0) \\
\mathbf{x}_2 & = & (0, 1, -1, 0, 0) \\ 
\mathbf{x}_3 & = & (0, 0, 0, 1, -1) \\
\mathbf{x}_4 & = & (0, 1, 0, 0, -1) \\
\mathbf{x}_5 & = & (0, 1, 0, -1, 0)
\end{array}
$$

es posible obtener las probabilidades $P\left[a_{i} = k \right]$ para $k \in \{-1, 0, 1\}$, basta calcular las frecuencias relativas de cada entrada. Así, en este caso, para las tres primeras entradas se tiene:

$$
\begin{array}{lcr}
P[a_1 = -1] = 0, & P[a_1 = 0] = \tfrac{4}{5}, & P[a_1 = 1] = \tfrac{1}{5}. \\ \\
P[a_2 = -1] = 0, & P[a_2 = 0] = \tfrac{2}{5}, & P[a_2 = 1] = \tfrac{3}{5}. \\ \\
P[a_3 = -1] = \tfrac{1}{4}, & P[a_3 = 0] = \tfrac{4}{5}, & P[a_3 = 1] = 0.
\end{array}
$$

Con estas probabilidades, es posible generar la población para la generación $n + 1.$
\end{ejemplo}

El algoritmo \ref{algo:etiquetado}, muestra el proceso para crear una nueva generación para el etiquetado de los datos.

\begin{algoritmo}[Algoritmo para etiquetado de datos]
\begin{tabbing}
\\ Sea $T$ el número de períodos en el conjunto de entrenamiento.
\\1. Inicializar la población $\{\mathbf{x}_i\}$ de tamaño $N$, cada $\mathbf{x}_i \in \{-1, 0, 1\}^{T}$. \\
2. \=Mientras no se cumpla criterio de paro: \\
 \>3. Seleccionar los mejores $M =\lfloor \tfrac{N}{2} \rfloor$ individuos.\\
 \>4. Con los $M$ individuos calcular $P[a_i = k]$ para $k \in \{-1, 0, 1\}$. \\
 \>5. \= Para $j = 1$ hasta $N$:\\
 \>\>6. Crear el individuo $\mathbf{x_j}$ de acuerdo a las probabilidades $P[a_i = k]$. \\

7. Regresa la nueva población. 
\end{tabbing}
\label{algo:etiquetado}
\end{algoritmo}

En la figura \ref{imagen:etiquetado} se observa que el proceso de etiquetado es capaz de localizar los mínimos y máximos locales presentes en el conjunto de entrenamiento.

\begin{figure}[ht]
\centering
\scalebox{0.8}{\includegraphics[width=1\linewidth]{imagenes/etiquetado.jpeg}}
\caption{\label{imagen:etiquetado} Resultado del proceso de etiquetado}
\end{figure}

\section{Atributos}
\label{seccion:atributos}
Los atributos usados en cada uno de los experimentos, son el conjunto de indicadores técnicos descritos en la sección \ref{subseccion:indicadores tecnicos}. Los parámetros para cada uno de ellos se describen a continuación, estos parámetros son los recomendados en la literatura del análisis técnico \cite{encycoplediaTechnicalIndicators}, \cite{technicalAnalysisKirkPatrick}.

\begin{itemize}
\item Oscilador Aroon con una ventana de tiempo de 25 días.

\item RSI con una ventana de tiempo de 14 días.

\item MFI con una ventana de tiempo de 14 días.

\item Williams \%R con una ventana de tiempo de 14 días.

\item CCI con una ventana de tiempo de 20 días.
\end{itemize}

\subsection{Discretización de los atributos}
\label{subseccion:discretizacion de atributos}
Ya que los algoritmos AQ y CN2 únicamente pueden trabajar con atributos discretos, es necesario emplear algún esquema de discretización para los indicadores técnicos utilizados.

De acuerdo a \cite{dataMiningUsingR}, un esquema de discretización para una variable $x$, determina (utilizando los datos de un conjunto de entrenamiento) un conjunto de puntos de corte, $b_1, b_2, \ldots, b_{k- 1}$ tales que:

$$
a(x) = 
\begin{cases}
v_1 & \mbox{Si } x \in (-\infty, b_1] \\
v_2 & \mbox{Si } x \in (b_1, b_2] \\
\ldots \\
v_{k - 1} & \mbox{Si } x \in (b_{k - 2}, b_{k -1}]\\
v_k & \mbox{Si } x \in (b_{k - 1}, \infty)\\
\end{cases}
$$
en donde $a$ es el atributo discretizado y los valores $v_i$ son los posibles valores que se asignan a este atributo.

Los esquemas de discretización explorados en este trabajo son:

\begin{itemize}
\item Discretización por intervalos con la misma longitud.
\item Discretización por cuantiles.
\end{itemize}

\subsubsection{Discretización por intervalos con la misma longitud}
Como su nombre lo señala, este tipo de discretización crea intervalos con una longitud constante. Para calcular cada intervalo, es necesario determinar el valor máximo ($M$) y el valor mínimo ($m$) que el atributo $x$ toma sobre el conjunto de entrenamiento $T$. A partir de estos valores es posible determinar el rango del atributo, definido como:

$$
\Delta_{T}^{k}(x) = \dfrac{M-m}{k}
$$
en donde $k$ es un entero que indica el número de intervalos deseados. El $i-$ésimo punto de corte está dado por
$$
b_i = m + i \Delta_{T}^{k}(x)
$$
para $i \in \{1, 2, \ldots, k - 1 \}$.

\subsubsection{Discretización por cuantiles}
Una de las principales deficiencias en la discretización por intervalos con la misma longitud, es que este esquema ignora la distribución de los valores del atributo, la cual puede ser no uniforme o incluso asimétrica.

Con el fin de considerar la distribución de los valores de un atributo, el esquema de discretización por cuantiles utiliza los cuantiles de la distribución de $x$ sobre el conjunto de entrenamiento. Así, si $k$ denota el número de intervalos deseados, entonces los puntos de corte están dados por
$$
b_i = q_{x}^{T} \left(\frac{i}{k} \right)
$$
en donde $i \in \{1, 2, \ldots, k - 1 \}$ y $q_{x}^{T} \left(y \right)$ es el cuantil $y$ de la distribución de los datos de la variable $x$ sobre el conjunto de entrenamiento $T$.

De esta manera, cada intervalo tendrá la misma área (probabilidad). Las figuras \ref{imagen:discretizacion intervalos} y \ref{imagen:discretizacion cuantiles} ejemplifican los esquemas de discretización para el caso $k = 8$.

\begin{figure}[ht]
\centering
\scalebox{0.7}{\includegraphics[width=1\linewidth]{imagenes/discretizacion-intervalos.png}}
\caption{\label{imagen:discretizacion intervalos} Ejemplo de la discretización por intervalos con la misma longitud}
\end{figure}

\begin{figure}[ht]
\centering
\scalebox{0.79}{\includegraphics[width=1\linewidth]{imagenes/discretizacion-cuantiles.png}}
\caption{\label{imagen:discretizacion cuantiles} Ejemplo de la discretización por cuantiles}
\end{figure}


\section{Aprendizaje incremental}
\label{seccion:aprendizaje incremental}
Tanto como para el algoritmo AQ como para el algoritmo CN2 se realizan dos tipos de experimentos. El primero de ellos está basado en un enfoque de \textit{aprender-aplicar-descartar}, en el cual, dado un conjunto de entrenamiento, se induce un conjunto de reglas las cuales se aplican al conjunto de prueba más cercano (el cual se convertirá en un nuevo conjunto de prueba). Después de esto, las reglas aprendidas se descartan (mecanismo de olvido).

La figura \ref{imagen:aprendizaje no incremental} ilustra este tipo de aprendizaje.
\begin{figure}[ht]
\centering
\scalebox{0.9}{\includegraphics[width=1\linewidth]{imagenes/aprendizaje-no-incremental.png}}
\caption{\label{imagen:aprendizaje no incremental} Aprendizaje no incremental}
\end{figure}

El otro experimento se basa en un \textit{aprendizaje incremental}, en este caso, se aprende un conjunto de reglas y en lugar de descartarlas, se ordenan de acuerdo a su desempeño sobre el conjunto de entrenamiento del cual fueron inducidas por primera vez, así como su desempeño sobre los conjuntos de prueba en los cuales han sido aplicadas. En este enfoque se mantiene un conjunto de $k$-mejores reglas de compra y $k$-mejores reglas de venta.

Para medir el desempeño de las reglas sobre los conjuntos de entrenamiento de los cuales se inducen por primera vez, se usa la suma de su \textit{soporte} y su \textit{exactitud de Laplace}.

El soporte de una regla se refiere al número de ejemplos cubiertos y que pertenecen a la clase positiva.

La exactitud de Laplace está dada por la ecuación (\ref{eqn:exactitud laplace})

\begin{equation}\label{eqn:exactitud laplace}
Exactitud\,de\,Laplace = \dfrac{ n_{c} + 1 }{ n_{tot} + k }
\end{equation}

en donde $n_c$ es el número de ejemplos de la clase $c$, que son cubiertos por la regla para esta misma clase, $n_{tot}$ es el número de ejemplos cubiertos por la regla y $k$ es el número de clases\footnote{Si una regla es consistente, entonces $n_c = n_{tot}$}.

Como se señala en \cite{CN2Improvements}, la exactitud de Laplace favorece el uso de reglas más generales sobre aquellas que son más específicas (elegidas al utilizar la exactitud clásica, $n_{c} / n_{tot} $). El siguiente ejemplo aclara lo anterior para el caso con sólo dos clases:

\begin{ejemplo}
Se tienen dos reglas, $R1$ y $R2$, tales que:

\begin{itemize} 
\item Regla $R1$ cubre $1000$ ejemplos de la clase $C1$ y $1$ de la clase $C2$.

\item Regla $R2$ cubre $5$ ejemplos de la clase $C1$ y $0$ de la clase $C2$.
\end{itemize}

Para la regla $R1$, se tiene que la exactitud clásica, en la clase $C1$ es:

$$ \dfrac{n_{C1}}{n_{tot}} = \dfrac{1000}{1001} = 0.99 = 99\%.$$

Para la regla $R2$, esta misma cantidad es:

$$ \dfrac{n_{C1}}{n_{tot}} = \dfrac{5}{5} = 1 = 100\%.$$

en este caso, se da preferencia a la regla $R2$, a pesar de que la regla $R1$ es más general para determinar la clase $C1$.

En cambio, con la ecuación (\ref{eqn:exactitud laplace}) se tiene para la regla $R1$:

$$ \dfrac{n_{C1} + 1}{n_{tot} + k} = \dfrac{1000 + 1}{1001 + 2} = 0.99 = 99\%.$$

mientras que para la regla $R2$:

$$ \dfrac{n_{C1} + 1}{n_{tot} + k} = \dfrac{5 + 1}{5 + 2} = 0.86 = 86\%.$$

y por lo tanto, se elige la regla $R1$ como una regla para determinar la clase $C1$.

\end{ejemplo}

Para medir el desempeño de las reglas sobre los conjuntos de prueba, primero se agrupan en pares de \textit{compra-venta}. Estos pares de reglas se forman a partir de las señales de compra y venta generadas durante el periodo de prueba y son recompensadas (penalizadas) de acuerdo a la ganancia (pérdida) que generan.

Por ejemplo, en un caso en el cual la regla de compra, $R_{C}$, indica que en el período $t_1$ se debe de comprar, esta compra tendrá un precio de ejecución $P_{C}$. Con el transcurso del tiempo, la regla de venta, $R_{V}$, indica que en el período $t_2 > t_1$, se debe de vender; esta venta tendrá precio de ejecución $P_{V}$. Estas dos reglas forman un par \textit{compra-venta} y la recompensa (o penalización) que se le asigna a cada regla en el par $\left(R_C, R_V\right)$ está dada por su ganancia (o pérdida) porcentual, es decir:

\begin{equation}\label{eqn:recompensa reglas}
Recompensa(R_C, R_V) = \dfrac{P_V (1 - c)}{P_C (1 + c) } - 1
\end{equation}

en donde $c$ es el costo de transacción.

Finalmente, cada regla que no aparece en el conjunto de prueba que se está evaluando, se penaliza con la pérdida promedio sobre este mismo conjunto de prueba; esto se utiliza como un mecanismo para ir descartando reglas debido al paso del tiempo.

De acuerdo a lo anterior, el puntaje de una regla $R$, antes de evaluar el conjunto de prueba $M$, está dado por:

\begin{equation} \label{eqn:puntaje de una regla}
puntaje_{-M}(R) = soporte(R) + Laplace(R) + \sum Recompensas
\end{equation}

en la ecuación (\ref{eqn:puntaje de una regla}), el término $ \sum Recompensas$ hace referencia a la suma de recompensas o penalizaciones que ha obtenido la regla $R$ para cada conjunto de prueba anterior al conjunto $M$.

Con los elementos anteriores, es posible ordenar las reglas y por lo tanto acumular el aprendizaje, así como identificar reglas útiles y descartar aquellas que no han funcionado.

La imagen \ref{imagen:aprendizaje_incremental} muestra el esquema del aprendizaje incremental que se propone en este trabajo.


\begin{figure}[ht]
\centering
\scalebox{0.9}{\includegraphics[width=1\linewidth]{imagenes/aprendizaje-incremental-ver2.png}}
\caption{\label{imagen:aprendizaje_incremental} Aprendizaje incremental}
\end{figure}


\section{Límites para las señales de venta}
\label{seccion:limites ventas}
Además de las reglas inducidas por cada algoritmo, se utilizan dos reglas adicionales que tienen el objetivo de capturar el perfil de riesgo de los inversionistas.

Estas nuevas reglas establecen dos límites horizontales relativos a la compra más reciente, un límite superior (mercados a la alza) y un límite inferior (mercados a la baja). El primero de estos límites puede ser interpretado como el nivel de codicia o conformidad que tiene un inversionista, mientras que el segundo se interpreta como el nivel de aversión al riesgo (que nivel de pérdida se puede tolerar).

Cada límite se expresa como un porcentaje relativo al precio de compra más reciente, por ejemplo un límite superior de $3\%$ indica que sólo se vende cuando la variación (porcentual) entre el precio de ejecución y el precio de compra es al menos $3\%$, de manera análoga, un límite superior de $-3\%$ indica que sólo se vende cuando la variación entre el precio de ejecución y el precio de compra es menor o igual a $-3\%$.

Utilizando estos límites y las reglas inducidas por cada algoritmo, una venta se ejecuta en dos etapas, a la primera de ellas se le llama la etapa de la señal y a la segunda la etapa de confirmación.

En la etapa de la señal, para cada día $t$ (después del cierre del mercado), se revisan las reglas de venta. Si alguna regla indica que se debe de vender, entonces se continua con la etapa de confirmación.

En la etapa de confirmación, a lo largo del día de ejecución, $t_{ejec}$ (ver los supuestos de la sección \ref{sec:supuestos del mercado}), se revisa si la variación porcentual entre algún precio generado en este día, $P_{t_{ejec}}$ y el último precio al que se compró, $P_{C}^{ult}$, rebasa alguna de las bandas horizontales.

Con estas dos etapas, se ejecuta una venta en el día $t_{ejec}$ si alguna de las siguientes condiciones se cumple:

\begin{align}
R_{venta}(t) \wedge (Var(t_{ejec}, P_{t_{ejec}}, P_{C}^{ult} ) > LS) \label{eqn:Venta limite sup}\\
R_{venta}(t) \wedge (Var(t_{ejec}, P_{t_{ejec}}, P_{C}^{ult} ) < LI) \label{eqn:Venta limite inf}
\end{align}

en donde $R_{venta}(t)$ representa una regla de venta, la cual se generó con la información del día $t < t_{ejec}$. $LS>0$, es el nivel del límite superior, expresado como un porcentaje. $LI<0$ es el nivel del límite inferior expresado como un porcentaje y $Var(t_{ejec}, P_{t_{ejec}}, P_{C}^{ult} )$ representa la variación porcentual entre los precios generados a lo largo del día $t_{ejec}$, $P_{t_{ejec}}$, respecto al último precio de compra, $P_{C}^{ult}$. Esta variación se calcula de acuerdo a la ecuación (\ref{eqn:variacion bandas})

\begin{equation} \label{eqn:variacion bandas}
Var(t_{ejec}, P_{t_{ejec}}, P_{C}^{ult} ) = \dfrac{P_{t_{ejec}}} {P_{C}^{ult}} - 1
\end{equation}

La figura \ref{imagen:bandas horizontales} muestra un ejemplo de las bandas horizontales relativas a una señal de compra.

\begin{figure}[htbp]
\centering
\scalebox{0.79}{\includegraphics[width=1\linewidth]{imagenes/bandas_horizontales.jpeg}}
\caption{\label{imagen:bandas horizontales}Bandas horizontales para las señales de venta relativas a una señal de compra}
\end{figure}

\section{Supuestos con los que se opera en el mercado}
\label{sec:supuestos del mercado}
Para cada experimento, se consideran los siguientes supuestos del mercado:

\begin{itemize}
\item El precio de compra o venta para un día $t_{ejec}$, es el promedio entre el precio máximo y el precio mínimo de ese día.

\item Dada una señal de compra en el día $t$, la transacción se ejecuta en el día $t+1$ utilizando el precio de ejecución.

\item El costo de cada transacción se fija en $0.25\%$. Así para cada compra con precio de ejecución $P_{C}$ se paga un total de 
$$P_{C}(1 + 0.0025)$$
por cada acción comprada, es decir, si se compran $N$ acciones se paga:

$$N \times P_{C}(1 + 0.0025).$$

De la misma forma, por cada acción que se vende a un precio $P_{V}$, se recibe un total de 
$$P_{V}(1 - 0.0025).$$
o si se venden $N$ acciones, se recibe

$$N \times P_{V}(1 - 0.0025).$$

\item En cada señal de compra, se compran las acciones que sean posible comprar utilizando todo el capital disponible (considerando el costo de la transacción). Es decir, si el precio de compra es $P_{C}$ y se cuenta con un capital $M$, entonces se pueden comprar

$$ N = \Bigl\lfloor \dfrac{M}{ P_{C} \left(1 + 0.0025 \right) } \Bigr\rfloor$$
acciones, en donde $\Bigl\lfloor x \Bigr\rfloor$ es la función piso.
\item Para cada señal de venta, se venden todas las acciones que se tengan.

\item Sólo se puede vender cuando se tienen acciones, es decir, siempre ocurre primero una compra antes de una venta.

\item El límite superior para las señales de venta, $LS$, se fija con un valor de $0.035$ o $3.5\%$ relativo al último precio de compra.

\item El límite inferior para las señales de venta, $LI$, se fija con un valor de $-0.03$ o $-3.0\%$ relativo al último precio de compra.

\item El capital inicial es de $\$100,000$ unidades monetarias (pesos en el caso del NAFTRAC y dólares de Estados Unidos para el SPDR S\&P 500).
\end{itemize}

%=============== Resultados ================= %
\chapter[Capítulo \thechapter: Resultados experimentales]{Resultados experimentales}
\label{capitulo:resultados experimentales}
Utilizando la metodología descrita en el capítulo \ref{capitulo:solucion propuesta}, para cada TRAC (ver sección \ref{seccion:indices accionarios}) se realizan cuatro experimentos distintos. Estos experimentos exploran las dos alternativas distintas para discretizar los atributos continuos, así como las dos alternativas para el tipo de aprendizaje aplicado (incremental o no incremental).

Por lo tanto, tenemos los siguientes experimentos:
\begin{itemize}
\item Aprendizaje incremental con discretización por intervalos (AI-DI).

\item Aprendizaje no incremental con discretización por intervalos (AN-DI).

\item Aprendizaje incremental con discretización por cuantiles (AI-DC).

\item Aprendizaje no incremental con discretización por cuantiles (AN-DC).
\end{itemize}
Finalmente, para cada experimento que involucra aprendizaje incremental, se seleccionan las mejores 5 reglas de compra y las mejores 5 reglas de venta.

\section{Resultados para el SPDR S\&P 500}
\label{seccion:resultados sp500}
Las tablas \ref{tabla:AQ-SP500} y \ref{tabla:CN2-SP500}, muestran un resumen de los resultados obtenidos en los $30$ conjuntos de prueba con la información del SPDR S\&P 500.
Como se observa en la tabla \ref{tabla:AQ-SP500}, para el algortimo AQ, el aprendizaje incremental resulta superior al no incremental en ambos métodos de discretización, obteniendo un exceso de ganancia promedio positivo (aunque muy pequeño).
\begin{center}
\begin{table}[htbp]
\centering
\begin{tabular}{ccccc}
\hline
\textbf{AQ} & \textbf{AI-DI} & \textbf{AN-DI} & \textbf{AI-DC} & \textbf{AN-DC} \\
\hline
Exceso de ganancia promedio & $0.37\%$ & $-0.39\%$ & $0.20\%$ & $-1.02\%$ \\
Casos mejores que \textit{compra y espera} & $11$ & $14$ & $10$ & $13$  \\
Casos peores que \textit{compra y espera} & $19$ & $16$ & $20$ & $17$ \\
Exceso de ganancia máximo & $34.64\%$ & $22.17\%$ & $29.72\%$ & $20.3\%$ \\
Exceso de ganancia mínimo & $-11.23\%$ & $-12.94\%$ & $-3.86\%$ & $-11.47\%$ \\
Ganancia monetaria promedio & $\$2,384.18$ & $\$1,620.11$ & $\$2,214.84$ & $\$ 994.96$ \\
\hline
\end{tabular}
\caption{\label{tabla:AQ-SP500} Resultados del algoritmo AQ para el SPDR S\&P 500}
\end{table}
\end{center}


\begin{center}
\begin{table}[htbp]
\centering
\begin{tabular}{ccccc}
\hline
\textbf{CN2} & \textbf{AI-DI} & \textbf{AN-DI} & \textbf{AI-DC} & \textbf{AN-DC} \\
\hline
Exceso de ganancia promedio & $-1.22\%$ & $-1.07\%$ & $-0.54\%$ & $-1.41\%$ \\
Casos mejores que \textit{compra y espera} & $11$ & $10$ & $12$ & $8$  \\
Casos peores que \textit{compra y espera} & $19$ & $20$ & $18$ & $22$ \\
Exceso de ganancia máximo & $15.69\%$ & $12.32\%$ & $11.81\%$ & $6.05\%$ \\
Exceso de ganancia mínimo & $-12.87\%$ & $-9.43\%$ & $-6.48\%$ & $-8.33\%$ \\
Ganancia monetaria promedio & $\$792.43$ & $\$946.07$ & $\$1,475.04$ & $\$ 604.99$ \\

\hline
\end{tabular}
\caption{\label{tabla:CN2-SP500}Resultados del algoritmo CN2 para el SPDR S\&P 500}
\end{table}
\end{center}
Por otro lado, como se muestra en la tabla \ref{tabla:CN2-SP500}, el algoritmo CN2 se desempeña pobremente en todos los experimentos, aunque se obtiene el mejor resultado utilizando un algoritmo de aprendizaje incremental.

\begin{figure}[htbp]
\centering
\scalebox{0.8}{\includegraphics[width=1\linewidth]{imagenes/sp500-2008-05-12.jpeg}}
\caption{\label{imagen:sp500-2008-05-12} Gráfica del precio del SPDR S\&P 500, 2008/05/12 - 2008/09/17}
\end{figure}

Es importante señalar que se consideran dos periodos correspondientes a la crisis financiera de 2008\footnote{Crisis hipotecaria en Estados Unidos de 2008.}. Como se ilustra en las figuras \ref{imagen:sp500-2008-05-12} y \ref{imagen:sp500-2008-09-18}, durante estos periodos el precio mostró una fuerte tendencia a la baja, en consecuencia, superar la ganancia de la estrategia \textit{compra y espera} es más fácil en este tipo de escenarios, como se puede constatar analizando las tablas \ref{tabla:AQ-SP500-2008} y \ref{tabla:CN2-SP500-2008}. Con estas tablas y las tablas \ref{tabla:AQ-SP500} y \ref{tabla:CN2-SP500}, se comprueba que, en la mayoría de los casos, el exceso de ganancia máximo corresponde precisamente a estos periodos.

Así pues, como se señala en \cite{Lohpetch2010}, el verdadero desafío se encuentra en superar a la estrategia \textit{compra y espera} en mercados con una tendencia a la alza. Por lo tanto, se calcula el desempeño de cada algoritmo únicamente en periodos en donde se tiene una tendencia creciente en los precios. En total se detectan 16 periodos de prueba con dicha característica, los resultados se muestran en las tablas \ref{tabla:AQ-SP500-alza} y \ref{tabla:CN2-SP500-alza}.

De acuerdo a estas tablas, en ningún caso los algoritmos fueron capaces de vencer a la estrategia \textit{compra y espera}. Sin embargo, nuevamente para el algoritmo AQ los mejores resultados se obtienen con un aprendizaje incremental.
\begin{figure}[htbp]
\centering
\scalebox{0.8}{\includegraphics[width=1\linewidth]{imagenes/sp500-2008-09-18.jpeg}}
\caption{\label{imagen:sp500-2008-09-18} Gráfica del precio del SPDR S\&P 500, 2008/09/18 - 2009/01/27}
\end{figure}

\begin{center}
\begin{table*}[htbp]
\centering
\begin{tabular}{ccccc}
\hline
\textbf{AQ periodo 2008} & \textbf{AI-DI} & \textbf{AN-DI} & \textbf{AI-DC} & \textbf{AN-DC} \\
\hline
$2008/05/12$ - $2008/09/17$ & $8.13\%$ & $2.33\%$ & $6.14\%$ & $3.57\%$ \\
$2008/09/18$ - $2009/01/27$ & $34.64\%$ & $22.17\%$ & $29.72\%$ & $20.37	\%$  \\
\hline
\end{tabular}
\caption{\label{tabla:AQ-SP500-2008} Exceso de ganancia del algoritmo AQ en la crisis financiera de 2008}
\end{table*}
\end{center}

\begin{center}
\begin{table*}[htbp]
\centering
\begin{tabular}{ccccc}
\hline
\textbf{CN2 periodo 2008} & \textbf{AI-DI} & \textbf{AN-DI} & \textbf{AI-DC} & \textbf{AN-DC} \\
\hline
$2008/05/12$ - $2008/09/17$ & $0.27\%$ & $-0.38\%$ & $5.06\%$ & $-2.81\%$ \\
$2008/09/18$ - $2009/01/27$ & $15.69\%$ & $12.32\%$ & $11.81\%$ & $-0.62\%$  \\
\hline
\end{tabular}
\caption{\label{tabla:CN2-SP500-2008} Exceso de ganancia del algoritmo CN2 en la crisis financiera de 2008}
\end{table*}
\end{center}

\begin{center}
\begin{table}[htbp]
\centering
\begin{tabular}{ccccc}
\hline
\textbf{AQ periodos a la alza} & \textbf{AI-DI} & \textbf{AN-DI} & \textbf{AI-DC} & \textbf{AN-DC} \\
\hline
Exceso de ganancia promedio & $-0.85\%$ & $-2.07\%$ & $-0.59\%$ & $-2.39\%$ \\
Casos mejores que \textit{compra y espera} & $5$ & $5$ & $6$ & $5$  \\
Casos peores que \textit{compra y espera} & $11$ & $11$ & $10$ & $11$ \\
Exceso de ganancia máximo & $2.33\%$ & $3.86\%$ & $2.30\%$ & $0.85\%$ \\
Exceso de ganancia mínimo & $-3.83\%$ & $-12.94\%$ & $-3.85\%$ & $-11.47\%$ \\
Ganancia monetaria promedio & $\$7,051.72$ & $\$5,830.17$ & $\$7,312.90$ & $\$5,514.90$ \\
\hline
\end{tabular}
\caption{\label{tabla:AQ-SP500-alza}Resultados del algoritmo AQ para el SPDR S\&P 500 en periodos a la alza}
\end{table}
\end{center}


\begin{center}
\begin{table}[htbp]
\centering
\begin{tabular}{ccccc}
\hline
\textbf{CN2 periodos a la alza} & \textbf{AI-DI} & \textbf{AN-DI} & \textbf{AI-DC} & \textbf{AN-DC} \\
\hline
Exceso de ganancia promedio & $-2.55\%$ & $-1.84\%$ & $-1.28\%$ & $-2.37\%$ \\
Casos mejores que \textit{compra y espera} & $5$ & $6$ & $6$ & $3$  \\
Casos peores que \textit{compra y espera} & $11$ & $10$ & $10$ & $13$ \\
Exceso de ganancia máximo & $0.52\%$ & $9.52\%$ & $1.00\%$ & $0.80\%$ \\
Exceso de ganancia mínimo & $-12.87\%$ & $-9.43\%$ & $-6.48\%$ & $-8.33\%$ \\
Ganancia monetaria promedio & $\$5,346.94$ & $\$6,059.76$ & $\$6,624.75$ & $\$5,529.46$ \\
\hline
\end{tabular}
\caption{\label{tabla:CN2-SP500-alza}Resultados del algoritmo CN2 para el SPDR S\&P 500 en periodos a la alza}
\end{table}
\end{center}

\section{Resultados para el NAFTRAC}
\label{seccion:resultados NAFTRAC}
Con los 15 conjuntos de prueba que contienen la información del NAFTRAC, se obtienen los resultados de las tablas  \ref{tabla:AQ-NAFTRAC} y \ref{tabla:CN2-NAFTRAC}.

De la misma forma que en los experimentos anteriores, el algoritmo AQ con aprendizaje incremental obtiene los mejores resultados para cada método de discretización, mientras que el algoritmo CN2 nuevamente presenta un desempeño pobre.

Si se analizan sólo los periodos a la alza (hay un total de 7 periodos con esta tendencia), de acuerdo a las tablas \ref{tabla:AQ-NAFTRAC-alza} y \ref{tabla:CN2-NAFTRAC-alza}, ambos algoritmos fueron capaces de vencer a la estrategia \textit{compra y espera} en 3 de 4 experimentos, obteniéndose el mejor resultado con el algoritmo AQ utilizando un aprendizaje incremental y una discretización basada en cuantiles.

\begin{center}
\begin{table}[h]
\centering
\begin{tabular}{ccccc}
\hline
\textbf{AQ} & \textbf{AI-DI} & \textbf{AN-DI} & \textbf{AI-DC} & \textbf{AN-DC} \\
\hline
Exceso de ganancia promedio & $1.29\%$ & $0.13\%$ & $0.59\%$ & $-0.30\%$ \\
Casos mejores que \textit{compra y espera} & $11$ & $8$ & $9$ & $7$  \\
Casos peores que \textit{compra y espera} & $14$ & $7$ & $6$ & $8$ \\
Exceso de ganancia máximo & $6.74\%$ & $5.16\%$ & $4.60\%$ & $1.85\%$ \\
Exceso de ganancia mínimo & $-6.78\%$ & $-6.27\%$ & $-3.44\%$ & $-3.46\%$ \\
Ganancia monetaria promedio & $\$1,697.77$ & $\$530.62$ & $\$993.49$ & $\$103.29$ \\
\hline
\end{tabular}
\caption{\label{tabla:AQ-NAFTRAC} Resultados del algoritmo AQ para el NAFTRAC}
\end{table}
\end{center}

\begin{center}
\begin{table}[h]
\centering
\begin{tabular}{ccccc}
\hline
\textbf{CN2} & \textbf{AI-DI} & \textbf{AN-DI} & \textbf{AI-DC} & \textbf{AN-DC} \\
\hline
Exceso de ganancia promedio & $-0.68\%$ & $0.17\%$ & $0.20\%$ & $0.06\%$ \\
Casos mejores que \textit{compra y espera} & $7$ & $10$ & $8$ & $9$  \\
Casos peores que \textit{compra y espera} & $8$ & $5$ & $7$ & $6$ \\
Exceso de ganancia máxima & $3.41\%$ & $4.33\%$ & $4.60\%$ & $4.00\%$ \\
Exceso de ganancia mínimo & $-4.87\%$ & $-6.00\%$ & $-3.44\%$ & $-6.46\%$ \\
Ganancia monetaria promedio & $\$-273.29$ & $\$578.85$ & $\$602.01$ & $\$461.66$ \\
\hline
\end{tabular}
\caption{\label{tabla:CN2-NAFTRAC} Resultados del algoritmo CN2 para el NAFTRAC}
\end{table}
\end{center}

\begin{center}
\begin{table}[h]
\centering
\begin{tabular}{ccccc}
\hline
\textbf{AQ periodos a la alza} & \textbf{AI-DI} & \textbf{AN-DI} & \textbf{AI-DC} & \textbf{AN-DC} \\
\hline
Exceso de ganancia promedio & $0.18\%$ & $0.87\%$ & $1.54\%$ & $-0.07\%$ \\
Casos mejores que \textit{compra y espera} & $4$ & $4$ & $5$ & $5$  \\
Casos peores que \textit{compra y espera} & $3$ & $3$ & $2$ & $2$ \\
Exceso de ganancia máximo & $5.03\%$ & $5.16\%$ & $4.60\%$ & $1.39\%$ \\
Exceso de ganancia mínimo & $-6.78\%$ & $-2.68\%$ & $-2.86\%$ & $-3.20\%$ \\
Ganancia monetaria promedio & $\$5,201.07$ & $\$5,895.67$ & $\$6,562.47$ & $\$4,950.47$ \\
\hline
\end{tabular}
\caption{\label{tabla:AQ-NAFTRAC-alza} Resultados del algoritmo AQ para el NAFTRAC en periodos a la alza}
\end{table}
\end{center}

\begin{center}
\begin{table}[htbp]
\centering
\begin{tabular}{ccccc}
\hline
\textbf{CN2 periodos a la alza} & \textbf{AI-DI} & \textbf{AN-DI} & \textbf{AI-DC} & \textbf{AN-DC} \\
\hline
Exceso de ganancia promedio & $-0.42\%$ & $1.06\%$ & $0.27\%$ & $0.87\%$ \\
Casos mejores que \textit{compra y espera} & $4$ & $6$ & $4$ & $5$  \\
Casos peores que \textit{compra y espera} & $3$ & $1$ & $3$ & $2$ \\
Exceso de ganancia máximo & $2.98\%$ & $3.91\%$ & $4.60\%$ & $4.00\%$ \\
Exceso de ganancia mínimo & $-3.11\%$ & $-0.77\%$ & $-3.11\%$ & $-1.45\%$ \\
Ganancia monetaria promedio & $\$4,605.01$ & $\$6,079.97$ & $\$5,294.73$ & $\$5,892.17$ \\
\hline
\end{tabular}
\caption{\label{tabla:CN2-NAFTRAC-alza} Resultados del algoritmo CN2 para el NAFTRAC en periodos a la alza}
\end{table}
\end{center}

\section{Sentido financiero de las reglas}
\label{seccion:sentido financiero de las reglas}
Al analizar el conjunto de reglas obtenidas, se observa que las reglas inducidas cuyo antecedente está formado por una sola condición, tienen una interpretación financieramente clara. Estas reglas pueden considerarse como un refinamiento a aquellas reglas presentadas en la tabla \ref{tabla:resumen reglas indicadores}, la única diferencia es el umbral de decisión utilizado. 

Particularmente, resulta interesante el siguiente par de reglas:
$$
\mbox{Si } RSI \in \left(-\infty, 53.7\right], \mbox{ entonces compra}
$$
$$
\mbox{Si } CCI \in \left(-23, \infty \right], \mbox{ entonces vende}
$$
estas reglas, actuando de manera conjunta, generaron el mayor exceso de ganancia para los experimentos del NAFTRAC en períodos en donde el mercado presentó una tendencia a la alza. Además, estas reglas pertenecieron al top $5$ de las mejores reglas en $13$ de los $15$ conjuntos de prueba del NAFTRAC.

Por otra parte, se observa que conforme el antecedente de las reglas contiene un mayor número de condiciones, se presenta una ambigüedad (financieramente hablando), por ejemplo, para la regla
$$
\mbox{Si } MFI \in \left(68.4. \infty\right) \mbox{ y } \mbox{ Williams \%R} \in \left(-109, -92.1 \right], \mbox{ entonces vende}
$$
de acuerdo a la interpretación financiera, $MFI \in \left(68.4. \infty\right)$ es una señal de venta, en cambio $\mbox{ Williams \%R} \in \left(-109, -92.1 \right]$ se interpreta como una señal de compra, en consecuencia, este tipo de reglas tiene una semántica financiera poco clara.

%=============== Conclusión ================= %
\chapter[Capítulo \thechapter: Conclusiones y trabajo futuro]{Conclusiones y trabajo futuro}
\label{capitulo:conclusiones}
%Las reglas aprendidas tienen sentido financiero?
%Aprendizaje incremental no es apropiado para CN2 pero si para AQ
\section{Conclusiones}
\label{seccion:conclusiones}
De acuerdo a los resultados de la sección \ref{capitulo:resultados experimentales},tanto para el mercado mexicano como para el estadounidense, el algoritmo AQ con aprendizaje incremental, presenta mejores resultados que cualquier experimento utilizando el algoritmo CN2. Más aún, para el caso del mercado mexicano, este algoritmo logra vencer (en promedio) a la estrategia \textit{compra y espera}, aún en mercados que presentan una tendencia a la alza. 

Una posible explicación del pobre desempeño del algoritmo CN2, es el hecho de que este algoritmo, por diseño, induce un conjunto de reglas ordenadas. Al aplicar un aprendizaje incremental y penalizar el desempeño de las reglas, este orden "natural" se ve modificado. Además, en el caso del aprendizaje no incremental, las reglas llevan un orden implícito, revisando las reglas de venta únicamente cuando existe una compra previa.

Es importante señalar que, a pesar de que no siempre se vence a la estrategia \textit{compra y espera}, únicamente en un caso se obtiene una ganancia monetaria promedio negativa (algoritmo CN2 experimento AI-DI, tabla \ref{tabla:CN2-NAFTRAC}), en el resto de los experimentos, en promedio, no se pierde dinero, una característica claramente deseable para cualquier algoritmo de inversión.

Desde el punto de vista de la teoría económica, los resultados experimentales indican que el mercado mexicano es menos eficiente que el mercado estadounidense. Esto está en línea con lo que se expone en \cite{CFA2019}, en donde se señala que uno de los factores que contribuye a la eficiencia del mercado, es el número de participantes que hay en este (a mayor número de participantes, mayor eficiencia hay).

Finalmente, como se muestra en la sección \ref{seccion:sentido financiero de las reglas}, el hecho de utilizar una metodología simbólica e interpretable, permite revisar si el conjunto de reglas tiene una semántica con sentido financiero. En este trabajo, fue posible obtener reglas con una interpretación financiera que coincide con el conocimiento previo desarrollado por los analistas financieros.

\section{Trabajo futuro}
\label{seccion:trabajo futuro}
%¿Como establecer las bandas horizontales?
%Parámetros de los indicadores técnicos
%Medida de riesgo
%Otras formas de dar puntajes de las reglas
%Supuestos del precio de ejecución
%Otros atributos
%Incorporar información relativa a la versión semifuerte de la HME
Entre las posibles líneas de extensión de este trabajo, se encuentran:

\begin{itemize}
\item Incorporar otro tipo de información pública, esto es, extender este trabajo con el fin de vencer la versión semifuerte de la hipótesis del mercado eficiente.

\item Idear una metodología para saber la proporción de dinero a invertir en cada señal de compra y no invertir todo el capital disponible.

\item Investigar el uso de otros indicadores técnicos, así como los parámetros de estos.

\item Analizar el desempeño de las estrategias considerando alguna medida de riesgo utilizada las finanzas.

\item Investigar la posición óptima para los límites de las señales de venta.

\item Probar con datos en escalas de tiempo más finas, por ejemplo, datos de cada 5 minutos.
\end{itemize}


\bibliography{references}
\addcontentsline{toc}{chapter}{Referencias}

%no citados en antecedentes ya que se desvían de la metodología propuesta pero fueron consultados para obtener un panorama más ámplio
\nocite{Preen2010}
\nocite{Kuo2013}
\nocite{Wang2014}
\nocite{Hu2015} %artículo de revisión de la literatura


\end{document}